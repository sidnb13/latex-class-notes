\subsection{Permutations}

Function to make row exchanges. Elimination with row exchanges:

\[A=LU\implies PA=LU\]

Works for any invertible $A$.

\[P=\text{identity with reordered rows (exchanges)}\]

Count of possible reorderings ($n\times n$ permutations): $n! = n(n-1)\cdots 3(2)(1)$.

\[\boxed{P^{-1}=P^T\text{ and }P^TP=I}\]

Defining a transpose, or flip over diagonal:

\[(A^T)_{ij}=A_{ji}\]

For symmetric matrices, transpose does not cause change; $A^T=A$. If two rectangular matrices $R^T$ and $R$
give a square matrix, then $R^TR$ is always symmetric.

\[\boxed{(R^TR)^T=R^TR^{TT}=R^TR}\]

\subsection{Vector Spaces and Subspaces}

Examples: $\R^2$ is all vectors in 2D space, $x-y$ plane: $\begin{bmatrix}3\\2\end{bmatrix}$.
$\R^3$ is all vectors with 3 components. All combinations of vectors in $\R^n$ yield a result in that space $\R^n$.

\[\boxed{\R^n\text{ is all column vectors with $n$ components.}}\]

The origin exists to allow for scalar multiplication and addition of vectors. Every vector space has a $\tb{0}$.

\subsubsection{Subspaces}

If a vector space is defined as 1st quadrant in $\R^2$, then multiplying by a negative scalar $k$ 
removes the result from that space, so it is not \textbf{closed} under that operation, so this is not a vector space. 
Vector space must be closed under linear combinations. Thus, subspace in $\R^2$ is all multiples of that vector, a line and the line must go through $\tb{0}$.
Every subspace must contain $\tb{0}$.\newline

Subspaces of $\R^2$:
\begin{itemize}
    \item All of $\R^2$
    \item Any line through $\tb{0}_2$ or $L$
    \item Just $\tb{0}_2$ or $Z$
\end{itemize}

Similarly, for $\R^3$ can have $\R^3$, plane, line, $\tb{0}_3$.\newline

Given $A=\left[\begin{array}{ll}
    1 & 3 \\
    2 & 3 \\
    4 & 1\end{array}\right]$, all linear combinations of these columns form a subspace. 
This is called \textbf{column space}, $C(A)$. This would form a plane in $\R^3$. Thus, the column space is a subspace.