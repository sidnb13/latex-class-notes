\subsection{Diagonalization and Properties}

Diagonal matrix has entries not along the main diagonal be all 0. Example:

\[A=\begin{bmatrix}1&0&0\\0&3&0\\0&0&-1\\ \end{bmatrix}\]

Characteristic polynomial ends up being

\begin{align*}
    p_A(\lambda)&=\mbox{det}\begin{bmatrix}1-\lambda&0&0\\ 0&3-\lambda&0\\ 0&0&-1-\lambda\\ \end{bmatrix}\\
    p_A(\lambda)&=(1-\lambda)(3-\lambda)(-1-\lambda)
\end{align*}

\noindent
Matrix similar to diagonal matrix is \textbf{diagonalizable}. Thus, $A$ is diagonalizable if
there exists an invertible $S$ and diagonal matrix $D$ such that $\boxed{S^{-1}AS=D}$. 
Because it is known that if $A\sim B$, $p_A(\lambda)=p_B(\lambda)$, whenever $A$ is diagonalizable, the
eigenvalues of $A$ will be diagonal entries of any diagonal matrix $A$ is similar to.

\noindent
Also note that for a diagonal matrix $D$, $D^t$ consists of all the diagonal entries $\lambda_1^t, \lambda_2^t\cdots, \lambda_n^t$:

\[D^t=\begin{bmatrix}
    \lambda_1^t&0&0&0\\
    0&\lambda_2^t&0&0\\
    0&0&\ddots&0\\
    0&0&0&\lambda_n^t\\
\end{bmatrix}\]

It then follows that if $S^{-1}AS=D$:

\begin{align*}
    S^{-1}AS&=D\\
    (S^{-1}AS)^t&=D^t\\
    (S^{-1}AS)(S^{-1}AS)(S^{-1}AS)\dots(S^{-1}AS)&=D^t\\
    S^{-1}A^tS&=D\\
    A^t&=SD^tS^{-1}\\
\end{align*}

\subsection{Diagonalization and Eigenbasis}

\textbf{A square matrix is diagonalizable if it has an eigenbasis.}

\noindent
Let the eigenbasis be $\mathfrak{E}=\left\{\tb v_1, \tb v_2, \tb v_3, \dots, \tb v_n\right\}$.
To show $A$ is diagonalizable, the $\mathfrak{E}$-matrix can be shown to be diagonal. First, let
$A$'s transformation be $T_A(\tb{x})=A\tb{x}$. The $\mathfrak{E}$-matrix is $D$. Thus, $\boxed{D[\tb x]_\mathfrak{E}=[T_A(\tb x)]_\mathfrak{E}}$.
The first column of $D$ is $[T_A(\tb{v}_1)]_\mathfrak{E}$. However, because $\mathfrak{E}$ is an eigenbasis for $A$,
it must hold true that:

\[A\tb v_1=T_A(\tb v_1)=\lambda_1 \tb v_1\]

\noindent
Thus, first column of $D$ is $[\lambda_1\tb{v}_1]_\mathfrak{E}$. Since $\lambda _1 \tb v_1=\lambda_1 \tb v_2+0\tb v_2+0\tb v_3+\dots+0\tb v_n$
as it is part of an eigenbasis, $[\lambda _1 \tb v_1]_\mathfrak{E}=\begin{bmatrix}\lambda_1\\0\\0\\ \vdots \\ 0 \end{bmatrix}$.
If repeated for all vectors in $\mathfrak{E}$, it ends up being that:

\[D=\begin{bmatrix}\lambda_1&0&0&0\\
    0&\lambda_2&0&0\\
    0&0&\ddots&0\\
    0&0&0&\lambda_n
\end{bmatrix}\]

It must also hold true that if $A$ is diagonalizable, it has an eigenbasis. It is known that:

\[S^{-1}AS=D=\begin{bmatrix}\lambda_1&0&0&0\\0&\lambda_2&0&0\\ 0&0&\ddots&0\\ 0&0&0&\lambda_n \end{bmatrix}\]

so a basis in $\R^n$ consisting of eigenvectors must be found. This is found in the columns of $S$, the change of basis matrix.
The similarity formula can be rearranged to get $AS=SD$. Thus, for the first column of $S$, $S\tb{e}_1$:

\begin{align*}
    A(S\tb e_1)&=(AS)\tb e_1\\
    &=(SD)\tb e_1\\
    &=S(D \tb e_1)\\
    &=S(\lambda_1 \tb e_1)\\
    &=\lambda_1 (S\tb e_1)\\    
\end{align*}

\noindent
This is indeed a basis as $S$ is required to be invertible.