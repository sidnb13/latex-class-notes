\section{Homogeneous Differential Equations}

\subsection{Homogeneity}

\begin{definition}[Homogeneity of polynomials]
    Polynomials where all terms are of the same degree are homogeneous.
\end{definition}

Homoegeneity of functions is analogous to assigning physical dimensions (e.g. length) to all of the variables.
If the function has the length dimension to the $k$th power, then it is homogeneous of degree $k$.

\begin{example}
    If $x,y$ are lengths, then the following is homogeneous of degree 3.
    \begin{equation}
        f(x,y)=2y^3\exp(\frac{y}{z})-\frac{x^4}{x+3y}
    \end{equation}
\end{example}

Alternate definition also suffices for generality.

\begin{definition}[Homogeneous function]
    $f(x,y)$ is homogeneous of degree $k$ iff $f(\lambda x, \lambda y)=\lambda^kf(x,y)$.
\end{definition}

\begin{definition}[Alternate definition of homogeneity]
    If $f(x,y)$ can be rewritten as $f(\frac{y}{x})$ or $f(\frac{x}{y})$ then it is homogeneous.
\end{definition}

\subsection{Homogeneous Differential Equations}

\begin{corollary}[Homogeneous DEs]
    If $M(x,y)$ and $N(x,y)$ are homogeneous and of same degree, then $M(x,y)dx+N(x,y)dy=0$
    is a homogeneous DE.
\end{corollary}
    
\begin{corollary}[Homogeneous DEs]
    $M(x,y)/N(x,y)$ is homogeneous of degree 0.
\end{corollary}

\begin{corollary}[Homogeneous DEs]
    If $f(x,y)$ is homogeneous of degree 0 in $x,y$, then $f(x,y)$ is a function of $y/x$ alone.
\end{corollary}

The ratio $M/N$ is a function of $y/x$, so the above can be rewritten as

\begin{align}
    \frac{dy}{dx}+g(\frac{y}{x})&=0\\
    \frac{d}{dx}(vx)+g(v)&=\frac{dv}{dx}+v+g(v)=0
\end{align}
    
Can thus transform into SOV problem by substituting $y=vx$ or $x=vy$, where $v$ is a function of $y$ or $x$.
Then, substitute back $v=\frac{y}{x}$ to obtain a general solution.