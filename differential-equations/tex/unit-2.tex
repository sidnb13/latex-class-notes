\section{Homogeneous Differential Equations}

\subsection{Homogeneity}

\begin{definition}[Homogeneity of polynomials]
    Polynomials where all terms are of the same degree are homogeneous.
\end{definition}

Homoegeneity of functions is analogous to assigning physical dimensions (e.g. length) to all of the variables.
If the function has the length dimension to the $k$th power, then it is homogeneous of degree $k$.

\begin{example}
    If $x,y$ are lengths, then the following is homogeneous of degree 3.
    \begin{equation}
        f(x,y)=2y^3\exp(\frac{y}{z})-\frac{x^4}{x+3y}
    \end{equation}
\end{example}

Alternate definition also suffices for generality.

\begin{definition}[Homogeneous function]
    $f(x,y)$ is homogeneous of degree $k$ iff $f(\lambda x, \lambda y)=\lambda^kf(x,y)$.
\end{definition}

\begin{definition}[Alternate definition of homogeneity]
    If $f(x,y)$ can be rewritten as $f(\frac{y}{x})$ or $f(\frac{x}{y})$ then it is homogeneous.
\end{definition}

\subsection{Homogeneous Differential Equations}

\begin{corollary}[Homogeneous DEs]
    If $M(x,y)$ and $N(x,y)$ are homogeneous and of same degree, then $M(x,y)dx+N(x,y)dy=0$
    is a homogeneous DE.
\end{corollary}
    
\begin{corollary}[Homogeneous DEs]
    $M(x,y)/N(x,y)$ is homogeneous of degree 0.
\end{corollary}

\begin{corollary}[Homogeneous DEs]
    If $f(x,y)$ is homogeneous of degree 0 in $x,y$, then $f(x,y)$ is a function of $y/x$ alone.
\end{corollary}

The ratio $M/N$ is a function of $y/x$, so the above can be rewritten as

\begin{align}
    \frac{dy}{dx}+g(\frac{y}{x})&=0\\
    \frac{d}{dx}(vx)+g(v)&=\frac{dv}{dx}+v+g(v)=0
\end{align}
    
Can thus transform into SOV problem by substituting $y=vx$ or $x=vy$, where $v$ is a function of $y$ or $x$.
Then, substitute back $v=\frac{y}{x}$ to obtain a general solution.

\section{Exact Equations}

If there exists an equation of the form $A(x)dx+B(y)dy=0$, the solution is a function with differential $A(x)dx+B(y)dy$.
Idea works for equations of form

\begin{equation}
    dF=Mdx+Ndy.
\end{equation}

So, $F(x,y)=c\implies dF=0$ and

\begin{equation}
    Mdx+Ndy=0.
\end{equation}

If there's a function $F$ such that $Mdx+Ndy$ is the \textbf{total differential}
of $F$, then Eq. 5 is an \textit{exact equation} by definition. Can rewrite the total differential from the chain rule:

\begin{equation}
    dF=\frac{\partial F}{\partial x}dx+\frac{\partial F}{\partial y}dy.
\end{equation}

So $M=\frac{\partial F}{\partial x},N=\frac{\partial F}{\partial y}$. We can take 2nd derivative to show these are equal because the partials are continuous (Clairaut's theorem).

\begin{eqnarray}
    \frac{\partial M}{\partial y}=\frac{\partial^2 F}{\partial y\partial x}\\
    \frac{\partial N}{\partial x}=\frac{\partial^2 F}{\partial y\partial x}.
\end{eqnarray}

\begin{definition}[Exactness]
    \begin{equation}
        \frac{\partial M}{\partial y}=\frac{\partial N}{\partial x}.
    \end{equation}
\end{definition}

\begin{proof}
    Let $\phi(x,y)$ be a function where $\frac{\partial \phi}{\partial x}=M$. $\phi$ is the function you get from integrating $Mdx$ wrt $x$ and holding $y$. Then

    \begin{equation}
        \frac{\partial^2\phi}{\partial y\partial x}=\frac{\partial M}{\partial y}=\frac{\partial N}{\partial x}
    \end{equation}

    Integrating both sides wrt $x$:

    \begin{equation}
        \frac{\partial \phi}{\partial x}=N+B'(y)
    \end{equation}

    where $B'(y)$ is the integration constant. Let

    \begin{equation}
        F=\phi(x,y)-B(y)
    \end{equation}

    such that 

    \begin{align}
        dF&=\frac{\partial \phi}{\partial x}dx+\frac{\partial \phi}{\partial y}dy-B'(y)dy\\
        &=Mdx+\left[N+B'(y)\right]dy-B'(y)dy\\
        &=Mdx+Ndy
    \end{align}
\end{proof}

\begin{example} We have the DE
    \begin{equation}
        3x(xy-2)dx+(x^3+2y)dy=0.
    \end{equation}

    Then, 

    \begin{equation}
        \frac{\partial M}{\partial y}=3x^2,\frac{\partial N}{\partial y}=3x^2
    \end{equation}

    The DE is exact, and $F=c$ is the solution.

    \begin{eqnarray}
        \frac{\partial F}{\partial x}=M=3x^2y-6x\\
        \frac{\partial F}{\partial y}=N=x^3+2y
    \end{eqnarray}

    Try to find $F$ from 18, integrate both sides wrt $x$ with an integration constant $T(y)$.

    \begin{equation}
        F=x^3y-3x^2+T(y)
    \end{equation}

    Using Eq. 19, can can find $\frac{\partial F}{\partial y}$ from Eq. 20 and equate:

    \begin{equation}
        x^3+T'(y)=x^3+2y\implies T'(y)=2y
    \end{equation}

    Because $F=c$ is the I.C., can conclude 

    \begin{equation}
        T(y)=y^2
    \end{equation}

    Thus,

    \begin{equation}
        F=x^3y-3x^2+y^2\Leftrightarrow x^3y-3x^2+y^2=c
    \end{equation}
\end{example}

\subsection{Linear Equations of Order 1}

If an equation is not exact, can attempt to do so by multiplying DE by an integrating factor.

\begin{definition}[Linear DE of order 1]
    \begin{equation}
        A(x)\frac{dy}{dx}+B(x)y=C(x)
    \end{equation}
\end{definition}

Divide each side by $A(x)$ to obtain

\begin{equation}
    \frac{dy}{dx}+P(x)y=Q(x)
\end{equation}

Suppose there exists for Eq. 25 a I.F. $v(x)>0$. Then,

\begin{equation}
    v(x)\left[\frac{dy}{dx}+P(x)y\right]=v(x)Q(x)
\end{equation}

becomes exact, or of form $Mdx+Ndy=0$. Here,

\begin{eqnarray}
    M=vPy-vQ\\
    N=v
\end{eqnarray}

Because the requirement is $\frac{\partial M}{\partial y}=\frac{\partial N}{\partial x}$,

\begin{align}
    vP&=\frac{dv}{dx}\\
    Pdx&=\frac{dv}{v}\\
    \ln v&=\int Pdx\\
    v&=\exp(\int Pdx)
\end{align}

We can then multiply both sides of the DE by this I.F. One side of this eqn will be of the product rule form, the derivative of $y\exp (\int Pdx)$:

\begin{equation}
    \exp(\int Pdx)\frac{dy}{dx}+P\exp(\int Pdx)y=Q\exp(\int Pdx)
\end{equation}

\subsection{General Solution of a Linear Equation}

Given the original form

\begin{equation}
    \frac{dy}{dx}+P(x)y=Q(x)
\end{equation}

suppose $P$ and $Q$ are continuous on $x\in (a,b)$ and $x=x_0$ is such a number. $y=y_0$ satisfies the initial condition.
This sol'n satisfies Eq. 34 for all $x$ in the interval. Multiplying Eq. 34 by integrating factor $\exp(\int Pdx)$ gives

\begin{equation}
    yv=\int vQ\:dx + c
\end{equation}

Because $v\neq 0$,

\begin{equation}
    y=v^{-1}\int vQ\: dx + cv^{-1}
\end{equation}

Given any $x_0,y_0$ in the interval, can find $c$ s.t. the DE is satisfied.
Every eqn of above form will have $P,Q$ with common interval of continuity and a unique set of solutions with one I.C. obtained by using the integrating factor.
These solutions are unique, so any other method yields a solution that aligns with the general solution–all possible solutions satisfying the DE on $x\in (a,b)$.