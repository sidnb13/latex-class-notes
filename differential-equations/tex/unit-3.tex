\section{Linear Equations with Constant Equations}

\subsection{Auxiliary Equation}

Linear homogeneous DE with constant coefficients can be expressed as

\begin{equation}
    a_{0} \frac{d^{n} y}{d x^{n}}+a_{1} \frac{d^{n-1} y}{d x^{n-1}}+\cdots+a_{n-1} \frac{d y}{d x}+a_{n} y=0
\end{equation}

Can be written in the form

\begin{equation}
    f(D)y=0
\end{equation}

where $f(D)$ is a linear differential operator. If the algebraic eqn $f(m)=0$
then we know $f(D)e^{mx}=0\implies y=e^{mx}$ is a solution to the form above.
$f(m)=0$ is the auxiliary equation associated with the DE.
Since the DE is of order $n$, the auxiliary equation is of degree $n$ with
roots $m_1,\ldots,m_n$.

Thus we have $n$ solutions $y_1=\exp(m_1x),\ldots,y_n=\exp(m_nx)$ assuming
the roots are \textbf{real} and \textbf{distinct} are then \textbf{linearly independent}.
The general solution is thus

\begin{equation}
    y=c_1\exp(m_1x)+\cdots+c_n\exp(m_nx)
\end{equation}

with arbitrary constants $c_1,\ldots,c_n$.

\subsubsection{Derivation}

We can say that $y^{(k)}$ is the $k$th derivative of $y$, so say the general form is

$$
a_ny^{(n)}+a_{n-1}y^{(n-1)}+\cdots+a_1y^\prime +a_0y=0
$$

If we take $y=e^{rx}$, then observe $y^{(n)}=r^ne^{rx}$. So rewrite the general form as

\begin{align*}
    a_nr^ne^{rx}+a_{n-1}r^{n-1}e^{rx}+\cdots+a_1re^{rx}+a_0e^{rx}&=0\\
    a_nr^n+a_{n-1}r^{n-1}+\cdots+a_1r+a_0&=0
\end{align*}

Solving for the roots $r$ in this characteristic equation helps us obtain the general solution.

\subsection{Auxiliary Equation Repeated Roots}

Need method for obtaining $n$ linearly independent solutions for $n$ equal roots of auxiliary equation.
Suppose auxiliary equation $f(m)=0$ has $n$ roots $m_1=m_2=\cdots=m_n=b$.
Thus, the operator function $f(D)$ has a factor $(D-b)^n$. Want to find $n$ linearly independent $y$ for which $(D-b)^ny=0$.
Use the substitution $y_k=x^ke^{bx}$ such that

\begin{equation}
    (D-b)^n(x^ke^{bx})=0,\;k=0,1,2,\ldots,n-1
\end{equation}

The functions $y_k=x^ke^{bx}$ are linearly independent because the respective powers $x^0,\ldots,x^k$ are linearly independent.
So the general solution takes form

\begin{equation}
    y=c_1e^{bx}+c_2xe^{bx}+\cdots+c_nx^{n-1}e^{bx}
\end{equation}

\subsection{Complex roots for the auxiliary equation}

Complex roots typically come in conjugate pairs of the form $a\pm bi$.
So using the general form of the auxiliary solution and the Euler identity $e^{i\theta}=\sin\theta + i\cos\theta$ and the fact that $\cos(-x)=\cos x, \sin(-x)=-\sin x$, we find that

\begin{align}
    y_1&=c_1e^{(a+bi)x}=c_1e^{ax}e^{bxi}=c_1e^{ax}\left(\sin bx + i\cos bx\right)\\
    y_2&=c_2e^{(a-bi)x}=c_2e^{ax}e^{-bxi}=c_2e^{ax}\left(-\sin bx + i\cos bx\right)\\
    y_1+y_2&=c_1e^{ax}\left(\sin bx + i\cos bx\right)+c_2e^{ax}\left(-\sin bx + i\cos bx\right)\\
    &=e^{ax}(c_1-c_2)\sin bx + e^{ax}(c_1+c_2)i\cos bx\\
    &=c_3e^{ax}\sin bx + c_4e^{ax}\cos bx
\end{align}

\subsection{Linear Independence}

Linear independence of two functions $f_1,f_2$ is calculated with a determinant called the Wronskian $W(f_1,f_2)$.

\begin{equation}
    W(f_1,f_2)=\left|\begin{matrix}f_1&f_2\\f_1\,'&f_2\,'\end{matrix}\right|
\end{equation}

Let $f_1,f_2$ be differentiable on $[a,b]$. if $W(f_1,f_2)(t_0)\neq 0$ for some $t\in [a,b]$ then $f_1,f_2$ are linearly independent on $[a,b]$.
If they are linearly dependent, $W(f_1,f_2)(t)=0\;\forall t$.

\subsection{Review of rational roots, synthetic division}

For rational roots test, take factors of constant coeff. $c$ divided by factors of
leading coefficient $a$ to get possible roots.

\subsection{Undetermined Coefficients}



\subsection{Variation of Parameters}

We start with the general equation

\begin{equation}
    ay''+by'+cy=g(x)
\end{equation}

First, find complementary solution $y_c$ from $y=y_p+y_c$. This is simply the solution of the homogeneous equation

\begin{equation}
    ay''+by'+cy=0
\end{equation}

Then, find the particular solution $y_p=u_1y_1+u_2y_2$ using Cramer's rule. Define 

\begin{equation}
W=
\begin{array}{|cc|}
    y_1&y_2\\
    y_1'&y_2'
\end{array}\;\;
W_1=\begin{array}{|cc|}
    0&y_2\\
    g(x)'&y_2'
\end{array}\;\;
W_2=\begin{array}{|cc|}
    y_1&0\\
    y_1'&g(x)'
\end{array}
\end{equation}

When we set up the matrix-vector equation $\mathrm{A}\mathbf{x}=\mathbf{b}$, it corresponds to the following:

\begin{equation}
    \begin{bmatrix}
        y_1 & y_2\\
        y_1' & y_2'
    \end{bmatrix}
    \begin{bmatrix}
        u_1\\u_2
    \end{bmatrix}
    =\begin{bmatrix}
        0\\g(x)
    \end{bmatrix}
\end{equation}

Following Cramer's rule, to find the solutions $1,\ldots,n$, we create $A_n$ by replacing the $n$th row of $A$ with $b$.

Strategy is to find $u_1', u_2'$ then integrate to find the coefficients and thus the eventual solution.

\begin{eqnarray}
    u_1'=\frac{W_1}{W}\\
    u_2'=\frac{W_2}{W}
\end{eqnarray}

\subsection{Reduction of Order}

Method is derived from the general 2nd order linear DE. Useful when $y_p$ does not correspond to a template solution class.

\begin{equation}
    y''+py'+qy=R\label{original}
\end{equation}

Assume there exists a solution $y=y_1$ of the homogeneous equation (e.g. can pick a solution from $y_c$).

\begin{eqnarray}
    y''+py"+qy=0
\end{eqnarray}

We then introduce a dependent variable from $y=y_1v$ as $v$. It follows that

\begin{eqnarray}
    y'=y_1v'+y_1'v\\
    y''=y_1v''+2y_1'v'+y_1''v
\end{eqnarray}

So substituting into \ref{original}, we get

\begin{equation}
    y_{1} v^{\prime \prime}+2 y_{1}^{\prime} v^{\prime}+y_{1}^{\prime \prime} v+p y_{1} v^{\prime}+p y_{1}^{\prime} v+q y_{1} v=R
\end{equation}

This rearranges to

\begin{equation}
    y_{1} v^{\prime \prime}+\left(2 y_{1}^{\prime}+p y_{1}\right) v^{\prime}+\left(y_{1}^{\prime \prime}+p y_{1}^{\prime}+q y_{1}\right) v=R .
\end{equation}

We use the definition of the homogeneous equation, since $y=y_1$ is a solution of the above.

\begin{equation}
    y_1v''+(2y_1'+py_1)v'=R
\end{equation}

If we let $v'=w$ and solve for $w$, then integrate, it is simply a first order problem with integrating factor

\begin{equation}
    k(x)=\exp{\int 2y_1'+py_1 dx}
\end{equation}

\subsection{Spring Oscillation}

\subsubsection{Derivation and Amplitude Form}

\subsubsection{Solution Cases}