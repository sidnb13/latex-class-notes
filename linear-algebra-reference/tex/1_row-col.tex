\subsection{Row picture}

Involves viewing matrix as linear equations graphed on a line or plane. Take the example $A\vec{x}=\vec{b}$ below:

\[\begin{bmatrix}
        1&2&3\\
        3&4&5\\
        4&5&6
    \end{bmatrix}
    \begin{bmatrix}
        x\\y\\z
    \end{bmatrix}
    =
    \begin{bmatrix}
        8\\9\\10
    \end{bmatrix}
\]

This can be viewed as the following system:

\[
    \spalignsys{
    1x + 2y + 3z = 8 ;
    3x + 4y + 5z = 9 ;
     4 + 5y + 6z = 10
    }
\]

\subsection{Column picture}

Involves viewing this setup as a linear combination of column vectors. Take $A\vec{x}=\vec{b}$ again:

\[
    x\begin{bmatrix}
        1\\3\\4
    \end{bmatrix}
    +y\begin{bmatrix}
        2\\4\\5
    \end{bmatrix}
    +z\begin{bmatrix}
        3\\5\\6
    \end{bmatrix}
    =
    \begin{bmatrix}
        8\\9\\10
    \end{bmatrix}.
\]

\subsection{Visualization in Space and Solutions}

\subsubsection{2D space}

In $\R^2$, the equations form a line. Independent column vectors means infinite linear combinations of these to get
a set of $\vec{b}$ in $\R^2$. If one column vector is dependent on another, they are parallel and various
combinations of $\vec{b}$ are on a line.

\subsubsection{3D space}

The equations form a plane in $\R^3$. If column vectors independent, infinite linear combination of $\vec{b}$
exist in 3D space. If one vector is a scaled combination of another and the third is independent, then solutions lie
on a line. If all three are interdependent, the solution is on a line.
