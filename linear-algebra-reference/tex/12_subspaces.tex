\subsection{Image and Kernel}

If $T:\;\R^m\rightarrow \R^n$ then $\mathrm{Im}(T)\subset \R^n$ and
$\mathrm{ker}(T)\subset \R^m$ because the associated matrix $A$ is $n\times m$ in dimension.\newline

\noindent
Both are closed under linear combinations:
\begin{itemize}
    \item If $\vec{y_1},\vec{y_2}\in \mathrm{Im}(T)$ then $a \vec{y}_{1}+b \vec{y}_{2} \in \mathrm{Im}(T)$ as well
    \item If $\vec{x_1},\vec{x_2}\in \mathrm{Ker}(T)$ then $a\vec{x_1}+b\vec{x_2}\in \mathrm{Ker}(T)$ as well
\end{itemize}

\subsection{Subspaces}

Collection of vectors in $\R^n$ is called a subspace in $\R^n$ if collection is nonempty and closed under linear combinations.
Examples (and counterexamples):

\begin{itemize}
    \item $W=\left\{\left(\begin{array}{c}
        3 s \\
        2+5 s
        \end{array}\right) \mid s \in \mathbb{R}\right\} \subset \mathbb{R}^{2}$
        is not a subspace because $\vec{0}$ is not contained within the set, so not closed under linear combinations
    \item $W=\left\{\left(\begin{array}{l}
        x_{1} \\
        x_{2} \\
        x_{2}
        \end{array}\right) \mid 2 x_{1}+x_{2}-x_{3}=0\right\}$ is a subspace due to matrix representation and 
        the image of this matrix containing $\vec{0}$ due to $T(\vec{0}=\vec{0})$
\end{itemize}

\noindent
$\R^2$ is not a subspace of $\R^3$ because though a plane can be drawn in $\R^3$, its components will be of the form
$\begin{bmatrix}x\\ y\\k\end{bmatrix}$, where $k$ is fixed. Since $\R^3$ vectors always have 3 coordinates, they can't represent $\R^2$.
$\R^2$ can only be represented by $\R^2$ vectors. Thus, $R^n$ is not a subspace of $\R^{n+1}$.\\

\noindent
\textbf{\textit{Claim:}} Span of a set of vectors in $\R^n$ is a subspace of $\R^n$.\\
\textbf{\textit{Proof:}}\\
Let $S=\{\vec{v_1},\vec{v_2},\cdots\vec{v_m}\}$. Let $\vec{w},\vec{y}\in \mathrm{span}(S)$.
Thus, $\vec{w}=\sum c_i\vec{v_i}$ and $\vec{y}=\sum d_i\vec{v_i}$ where $d_i,c_i\in \R$.

\begin{align*}
    a\vec{w}+b\vec{y}&=a\sum c_i\vec{v_i}+b\sum d_i\vec{v_i}\\
    &=\sum a c_{i} \vec{v}_{i}+\sum b d_{i} \vec{v}_{i}\\
    &=\sum\left(a c_{i}+b d_{i}\right) \vec{v}_{i} \in \operatorname{span}(S)
\end{align*}

\noindent
List of subspaces in $\R^2$ would be $\R^2$, $\{t\vec{v}\mid t\in\R\}$, $\{\vec{0}\}$.

\subsection{Intersection and Union}

If $V$ and $W$ are collections of vectors in $\R^n$:
\begin{itemize}
    \item $V \cap W=\{\vec{x} \mid \vec{x} \in V \text { and } \vec{x} \in W\}$ is the intersection
    \item $V \cup W=\{\vec{x} \mid \vec{x} \in V \text { or } \vec{x} \in W\}$ is the union
\end{itemize}

\subsection{Redundant Vectors}

If for some transformation $T$ there exists the following:

\[\operatorname{im}(T)=\operatorname{span}\left\{\left(\begin{array}{l}
    1 \\
    1 \\
    1
    \end{array}\right),\left(\begin{array}{l}
    2 \\
    2 \\
    2
    \end{array}\right),\left(\begin{array}{l}
    1 \\
    2 \\
    3
    \end{array}\right),\left(\begin{array}{l}
    2 \\
    3 \\
    4
    \end{array}\right)\right\}\]

There are redundant vectors in this case. The minimum number of vectors in the span is 2, 
for $\vec{0}$ cannot be produced then. With 3 vectors in $\R^3$, any one can be the result of linear combinations of the other 2.
So, it would be appropriate to say that:

\[\operatorname{im}(T)=\left\{\left(\begin{array}{l}
    1 \\
    1 \\
    1
    \end{array}\right),\left(\begin{array}{l}
    1 \\
    2 \\
    3
    \end{array}\right)\right\}\]

These are then \textbf{linearly independent}. This set forms a \textbf{basis} for that set of vectors.
Thus, the basis can be found for any matrix. The basis of $I_n$ is then $\{\vec{e}_1,\vec{e}_2\cdots \vec{e}_n\}$.

\subsection{Intersection and Union}

\begin{framed}
    \noindent If $V$ and $W$ are subspaces of $\R^n$, then $V\cap W$ 
    is a subspace of $\R^n$ and $V\cup W$ is \textbf{not} a subspace of $\R^n$.
\end{framed}

\noindent
An intersection is the items contained in both sets, so $\vec{0}\in V\cap W$.
If $\vec{v},\vec{w}\in V\cap W$, then $\vec{v},\vec{w}\in V$ and $\vec{v},\vec{w}\in W$. This means that
$\vec{v}+\vec{w}\in V$ and $\vec{v}+\vec{w}\in W$ so $\vec{v}+\vec{w}\in V\cap W$.
Similarly, if some $k\vec{v}\in V\cap W$ where $k\in \R$ then $k\vec{v}\in V$ and $k\vec{v}\in V$. 
Thus, $V\cap W$ is a subspace of $\R^n$.\\

\noindent
The union is the items contained in either set.
If $V=\mathrm{span}(\vec{e}_2)$ and $W=\mathrm{span}(\vec{e}_1)$, then let $\vec{y}=\begin{bmatrix}0\\ y\end{bmatrix}\in V$
and $\vec{x}=\begin{bmatrix}x\\ 0\end{bmatrix}\in W$. Thus, $\vec{y},\vec{x}\in V\cup W$. However,
$a\vec{x}+b\vec{y}\notin V\cup W$ where $a,b\in \R$.