\subsection{Rank and independence}

If $\{\vec{v}_1,\vec{v}_2,\cdots ,\vec{v}_m\}$ is a collection if independent vectors then

\[
\left(\begin{array}{ccccc}
\mid & \mid & \mid & & \mid \\
\vec{v}_{1} & \bar{v}_{2} & \bar{v}_{3} & \ldots & \bar{v}_{m} \\
\mid & \mid & \mid & & \mid
\end{array}\right)
\]

must have a rank of $m$. This is because row reducing the matrix corresponds to the following relation:

\[c_1\vec{v}_1+c_2\vec{v}_2+c_3\vec{v}_3+\dots +c_m\vec{v}_m=\vec{0}\]

\noindent
Also, $m\leq n$ where $n$ is the number of rows in each column vector, in order to have linear independence for this set.

\subsection{Dimension}

Considering an $xy$-plane in $\R^3$:

\[V=\left\{\begin{pmatrix}s\\t\\0\\ \end{pmatrix}\Bigg| s,t \in \mathbb{R} \right\}\]

The basis of this set contains 2 vectors (e.g. dimension of 2), with example being:

\[\mathfrak{B}=\left\{\begin{pmatrix}1\\0\\0 \end{pmatrix} , \begin{pmatrix} 0\\1\\0 \end{pmatrix} \right\}\]

\begin{framed}
    If $V$ is a subspace of $\R^n$ and $\mathfrak{B}$ and $\mathfrak{C}$ are two bases of $V$, then
    $\mathfrak{B}$ and $\mathfrak{C}$ contain the same number of vectors.
\end{framed}

\textbf{Dimension} of a subspace is number of vectors in the basis.

\subsubsection{Example}

Considering the following matrix:

\[A=\begin{pmatrix}
    1&2&0&1&2\\
    1&2&0&2&3\\
    1&2&0&3&4\\
    1&2&0&4&5\\
    \end{pmatrix}\]

By discounting the redundant vectors, a possible basis for $\mathrm{Im}(A)$:

\[\mathfrak{B}_\mathrm{image}=\left \{\begin{pmatrix} 1\\1\\1\\1 \end{pmatrix},\begin{pmatrix}1\\2\\3\\4 \end{pmatrix} \right\}\]

So the dimension of $\mathrm{Im}(A)$ is 2.
Finding a basis for $\mathrm{ker}(A)$ is the same as solving $A\vec{x}=\vec{0}$:

\[\mbox{ker}(A)=\left\{\begin{pmatrix}-2s-w\\ s\\t\\-w\\w \end{pmatrix} \Bigg| s,t,w\in\mathbb{R} \right\}=
\left\{s\begin{pmatrix}-2\\1\\0\\0\\0\\\end{pmatrix}+t\begin{pmatrix}0\\0\\1\\0\\0\\\end{pmatrix}+w\begin{pmatrix}-1\\0\\0\\-1\\1\\\end{pmatrix}\Bigg|s,t,w\in\mathbb{R}\right\}\]

So the basis for $\mathrm{ker}(A)$:

\[\mathfrak{B}_\mathrm{kernel}=\left\{\begin{pmatrix}-2\\1\\0\\0\\0\\ \end{pmatrix}, \begin{pmatrix} 0\\0\\1\\0\\0\\ \end{pmatrix}, \begin{pmatrix}-1\\0\\0\\-1\\1\\ \end{pmatrix} \right\}\]

And dimension of $\mathrm{ker}(A)$ is 3.
However, it is shown that $\mathrm{rref}(A)$ gives dimension of \textbf{image and kernel}.

\subsection{Rank-Nullity Theorem}

\begin{framed}
    \begin{itemize}
        \item If $T$ is a linear transformation, 
        then \\$\mathrm{dim}(\mathrm{Im}(T))+\mathrm{dim}(\mathrm{ker}(T))=\text{dimension of domain of }T$
        \item If $A$ is a matrix, then $\mathrm{rank}(A)+\mathrm{nullity}(A)=\text{number of columns of }A$
        \item In a linear system, 
        \\$\text{number of leading variables}+\text{number of free variables}=\text{total number of variables}$
    \end{itemize}
\end{framed}

Considering non-invertible matrices $A$ and $B$, let $AB$ be invertible.
It must hold true that $\mathrm{ker}(B)=\{\vec{0}\}$. If the dimensions of $B$ are $p\times n$,
$\mathrm{Im}(B)$ is a subspace of $\R^p$ has dimension $n$. This means that it is a vertically rectangular
matrix with $n\leq p$. Thus, $A$ is $n\times p$ so it is horizontally rectangular.