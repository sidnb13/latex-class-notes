\subsection{Introduction to Determinant}

Can define the $2\times 2$ determinant as a function $D:\mathbb{M}_{2\times2}\rightarrow R$.
It can be observed that $2\times 2$ matrix $A$ is only invertible if $D(A)=ad-bc\neq 0$.

\subsection{Cross-Product}

Given $\begin{bmatrix}a\\ c\end{bmatrix}$ and $\begin{bmatrix}b\\ d\end{bmatrix}$, the cross
product is defined as the $\R^3$ vector $D(A)\vec{e}_3=(ad-bc)\vec{e}_3$. The direction
of this vector is the sign of $\mathrm{det}(A)$.

\noindent
Can visualize using right hand rule: if sweeping index into middle is appropriate for the vectors,
then the direction of thumb is cross-product direction (positive). Otherwise, sign is negative.

\subsubsection{Algorithm}

For $\vec{v}=\begin{bmatrix}v_x\\ v_y\\ v_z\end{bmatrix}$ and $\vec{w}=\begin{bmatrix}w_x\\ w_y\\ w_z\end{bmatrix}$:

\[\boxed{\vec{v}\times\vec{w}=c_z\vec{e}_1+c_y\vec{e}_2+c_z\vec{e}_z}\]

Following through, to calculate each component ignore the desired row and perform cross-product on remaining matrix:

\[c_x=\begin{bmatrix}v_y\\ v_z\end{bmatrix}\times \begin{bmatrix}w_y\\ w_z\end{bmatrix}=v_yw_z-v_zw_z\]

The $y$ component is done as $bc-ad$ compared to $ad-bc$.

\[c_y=\begin{bmatrix}v_x\\ v_z\end{bmatrix}\times \begin{bmatrix}w_x\\ w_z\end{bmatrix}=w_xv_z-w_zv_x\]

\[c_z=\begin{bmatrix}v_y\\ v_z\end{bmatrix}\times \begin{bmatrix}w_y\\ w_z\end{bmatrix}=v_xw_y-v_yw_x\]

\subsection{Determinant Theory}

Considering $A=\begin{bmatrix}|&|&| \\\vec v_1 & \vec v_2 &\vec v_3 \\ |&|&| \\ \end{bmatrix}$,
say it is singular such that $\vec{v}_3\in \mathrm{span}\{\vec{v}_1,\vec{v}_2\}$. Because it is assumed that
$\vec{v}_1,\vec{v}_2$ are linearly independent, then $\mathrm{span}\{\vec{v}_1,\vec{v}_2\}$ is perpendicular to 
$\vec{v}_1\times \vec{v}_2$ by definition. Thus, $\boxed{(\vec{v}_1\times\vec{v}_2)\cdot \vec{v}_3=0}$.
If $\vec{v}_1,\vec{v}_2$ are not linearly independent, then this is still 0 because the cross-product (area of parallelogram made by vectors)
is still 0.

\noindent
If $\vec{v}_1,\vec{v}_2,\vec{v}_3$ are linearly independent, then $(\vec{v}_1\times\vec{v}_2)\cdot \vec{v}_3\neq0$.

\[\boxed{D(A)=\mathrm{det}(A)=(\vec{v}_1\times\vec{v}_2)\cdot \vec{v}_3}\]

\noindent
The sub-matrices used when computing cross-products are \textbf{minors}.
Can rewrite determinant:

\[\boxed{\det A= a_{1,3}\left|A_{1,3}\right|-a_{2,3}\left|A_{2,3}\right|+a_{3,3}\left|A_{3,3}\right|}\]

Must use following rule for the sign of constant terms $a_{m,n}$ (dot product):

\[\begin{bmatrix}+&-&+\\-&+&-\\+&-&+\end{bmatrix}\]

\subsection{Rules}

Determinant of $n\times n$ follows recursively:

\[\boxed{\mbox{det}A=a_{1,1}\left|A_{1,1}\right|-a_{1,2}\left|A_{1,2}\right|+a_{1,3}\left|A_{1,3}\right|+\dots\pm a_{1,n}\left|A_{1,n}\right|}\]

\noindent
Rules:
\begin{itemize}
    \item Swapping rows multiplies determinant by -1
    \item Multiplying row by $m$ scales determinant by $m$
    \item Replacing row with sum of row and multiple of another does not change determinant
    \item If $A$ and $B$ are $n\times n$, then $\mathrm{det}(AB)=\mathrm{det}(A)\mathrm{det}(B)$
    \item Cramer's rule: If $A\vec{x}=\vec{b}$ is a linear system with invertible $A$ then $\vec{x}$ components can be determined
    from $x_i=\frac{\mathrm{det}(A-_{b,i})}{\mathrm{det}(A)}$ where $A-_{b,i}$ replaces $i^{\mathrm{th}}$ column of $A$ with $\vec{b}$
\end{itemize}