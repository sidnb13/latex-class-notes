\subsection{Example with transformation}

The $2\times 2$ matrix $A$ with property $R_\theta(\vec{v})=A\vec{v}$ rotates the vector by $\theta$.
Using the unit circle to find the coordinates using the basis vectors $\vec{e}_1$ and $\vec{e}_2$:

\[
    R_{\theta}\left(\vec{e}_{1}\right)=\left[\begin{array}{l}
    \cos \theta \\
    \sin \theta
    \end{array}\right]
\]

\[
    R_{\theta}\left(\vec{e}_{2}\right)=\left[\begin{array}{c}
    -\sin \theta \\
    \cos \theta
    \end{array}\right]
\]

This results in $A$:

\[
    \left[\begin{array}{cc}
    \cos \theta & -\sin \theta \\
    \sin \theta & \cos \theta
    \end{array}\right]
\]

Finding the inverse of this is simply rotating back by $\theta$, so finding $R^{-1}_{\theta}$:

\[
    \left[\begin{array}{cc}
    \cos (-\theta) & -\sin (-\theta) \\
    \sin (-\theta) & \cos (-\theta)
    \end{array}\right]
    =
    \left[\begin{array}{cc}
    \cos \theta & \sin \theta \\
    -\sin \theta & \cos \theta
    \end{array}\right]
\]