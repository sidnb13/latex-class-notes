\subsection{Row and Column Swapping}

Can define elementary row operations in the identity matrix.

\subsubsection{Swapping rows}

\[
    \begin{bmatrix}
        0&1\\
        1&0
    \end{bmatrix}
    \begin{bmatrix}
        a&b\\
        c&d
    \end{bmatrix}
    =
    \begin{bmatrix}
        c&d\\
        a&b
    \end{bmatrix}
\]

Modifier $B$ is always on \textbf{left}.

\subsubsection{Swapping columns}

\[
    \begin{bmatrix}
        a&b\\
        c&d
    \end{bmatrix}
    \begin{bmatrix}
        0&1\\
        1&0
    \end{bmatrix}
    =
    \begin{bmatrix}
        b&a\\
        d&c
    \end{bmatrix}
\]

Modifier $B$ is always on \textbf{right}.

\subsection{Elimination steps}

Performing elimination:

\[E_{2,1}A+E_{3,2}A=(E_{2,1}E_{3,2})A=U\]

Elimination algorithm:
\begin{itemize}
    \item $E_{2,1}$ is the pivot. Swap with $R_2$ if 0 (and $(2,1)$ is nonzero).
    \item $E_{3,2}$ involves getting $(2,2)$ as a pivot assuming nonzero to get $(3,2)$ as 0
    \item Result is invertible and non-singular, where $U$ is upper-triangular
\end{itemize}

Matrix multiplication is not necessarily commutative but always associative.

\subsection{Matrix Multiplication Facts}

If $A$ is an $m\times n$ matrix and $B$ is $n\times p$, then $AB=C$ must be $m\times p$.
Standard method would be to take dot products by row and column. By column: Columns of $C$ are combinations of columns
of $A$. By row: Rows of $C$ are combinations of rows of $B$.

\subsection{Example (Row)}

\[
    \left[\begin{array}{ll}
        2 & 7 \\
        3 & 8 \\
        4 & 9
        \end{array}\right]\left[\begin{array}{ll}
        1 & 6 \\
        0 & 0
        \end{array}\right]=\left[\begin{array}{l}
        2 \\
        3 \\
        4
        \end{array}\right]\left[\begin{array}{ll}
        1 & 6
        \end{array}\right]+\left[\begin{array}{l}
        7 \\
        8 \\
        9
        \end{array}\right]\left[\begin{array}{ll}
        0 & 0
    \end{array}\right]
\]