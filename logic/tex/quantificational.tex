\section{Quantifiers (Bonevac)}

\begin{itemize}
    \item Aristotle logic gives arguments restricted form. Every sentence is of form $\{\text{some,all,no}\}F\{\text{are,not}\}G$
    \item Syllogistic argument has two such setences as premises and one as a conclusion. The meshing of these is specific, limited.
    \item \ti{Sentential logic} takes sentences as basic analytical units, covers broader realm.
    \item Sentential logic does not solve problems of syllogistic, e.g. cannot explain why an argument is valid
    \item Divergence b/w syllogistic and sentential resolved by Friege and Peirce
    \begin{itemize}
        \item Introduced determiners (e.g. all, some, no, every, any, etc.)
        \item Universal quantifier $\forall$ and existential quantifier $\exists$
    \end{itemize}
\end{itemize}

\subsection{Constants and Quantifiers}

Atomic sentences consist of a main subject/noun phrase and verb phrase. Examples of verb phrases:

\begin{enumerate}
    \item is a man
    \item sleeps very soundly
    \item kicked the ball into the end zone
\end{enumerate}

Verb phrases are terms in syllogistic logic, are \tb{true or false} of individual objects. E.g. a man can sleep soundly or not.
Objects of which verb phrase \tb{satisfy} it, phrase will \tb{apply} to them.
The set of objects which make a verb phrase true are called \tb{extensions}.

Verb phrases combine with noun phrases to form \ti{sentences}. Noun phrases specify an object or groups of objects, since verb phrases describe their truth values.
Following examples are complete sentences.

\begin{itemize}
    \item Socrates is a man
    \item Mr. Hendley sleeps very soundly
    \item Nate have Fred a copy of the letter
\end{itemize}

Upper-case alphabet letters w/wout subscripts are \tb{predicates}. Each predicate has an assigned number.
Predicate with number $n$ is $n$-ary. Predicate yields a truth value when combined with certain number of objects.
Assigned value $n$ indicates of how many objects the predicate takes on this truth value.

For example, man has singulary predicates (true false of a single object). Binary predicates take on two objects. Example is Person 1 respects Person 2, but Person 1 respects makes no sense.

To structure sentences, take example.

\begin{example}
    Something is missing
\end{example}

Missing applies to an object.

\begin{example}
    Missing(something)
\end{example}

We use lowercase letters to denote constants, capital for predicates as discussed.
Socrates is a man can be translated to $Ma$ where $a$ is a constant symbolizing Socrates, $M$ means man.

\begin{example} Translation\\
    (for some $x$)($x$ is missing)\\
    $\exists x M x$
\end{example}

\section{Quantificational Logic}

Main idea is that we use predicates to describe properties of variables, which we say are \ti{quantified}.
Say we define $Lxy$ to be that $x$ likes $y$. $L$ is our predicate, and $x,y$ are variables, of which we can pass any number to an (appropriate) predicate.

Idea of quantifiers
\begin{itemize}
    \item $\exists x$: there is some individual $x$ such that
    \item $\forall x$: every individual $x$ such that
\end{itemize}

\subsection{Subtlety with quantifier order}

Difference between
\begin{gather*}
    1.\;\exists y\forall x (Lxy)\\
    2.\;\forall x\exists y(Lxy)
\end{gather*}

\ti{First}: there exists someone who likes all individuals.\\
\ti{Second}: for every individual, there exists someone who likes them. 

\subsection{Rules and properties}

An identity predicate exists to test equality of two subjects.

\begin{definition} (Identity predicate)
    $$x=y:\text{$x$ is equal to $y$}$$
\end{definition}

We can obviously tell that identical subjects have the same properties, so

\begin{definition} (Leibniz' law)
    $$\forall x\forall y(x=y\then (Px\leftrightarrow Py))$$
\end{definition}

We can't ``for loop'' over all properties of subjects, so that must be done manually.

\subsection{Describing quantities}

Let $Px$ mean that $x$ has property $P$. Be aware that we can trivially extend the below equivalences to $n$ subjects.

\begin{tabular}{l|l}
    Statement & QL\\
    \hline
    \hline
    There is exactly one thing that is $P$ & $\exists x(Px\land \forall y(Py\then x=y))$\\
    There are exactly two things that are $P$ & $\exists x\exists y(Px\land Py\land x\neq y\land \forall z(Pz\then (z=x\lor z=y))$\\
    There are at least two things that are $P$ & $\exists x\exists y(Px\land Py\land x\neq y)$\\
    There are less than two things that are $P$ & $\forall x\forall y(\lnot Px\lor \lnot Py\lor x=y)$ (this is not identical to the first)\\ 
\end{tabular}

\subsection{Interpretations}

To make interpretations of sentences (i.e. constant values using names) we set $UD=\{a,b,c\}$ or whatever,
then interpret the predicates with these names $Pa,Pb,Pc$, etc. Here we can cherry pick $a,b,c$ to arbitrarily give truth values to $P$ when interpreted. Useful for informal proofs.

\subsection{Natural deduction}

Below is a list of inferences or tautologies for natural deduction.

\begin{tabular}{l|l|l}
    Name & Tautology or Inference & Code\\
    \hline
    \hline
    Quantifier Exchange (occurs both ways) & $\exists x(\dots x \dots)\leftrightarrow \forall x \lnot(\dots x\dots)$ & QEx\\
    Universal Instantiation & $\forall x(\dots x\dots)\therefore \dots a \dots$ where $a$ is a new name in proof & UI\\
    Universal Generalization & $\dots a \dots\; \therefore\forall x(\dots x\dots)$ iff $a$ was declared by UI & UG\\
    Existential Instantiation & $\exists x(\dots x\dots)\therefore \dots a\dots$ where $a$ is any name & EI\\
    Existential Generalization & $\dots a\dots \;\therefore\exists x(\dots x\dots)$ & EG\\
    Leibniz Law & $\forall x\forall y(x=y\then (Px\leftrightarrow Py))$ & Leibniz
\end{tabular}