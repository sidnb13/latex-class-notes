\subsection{Separation of Variables}\label{subsec:separation-of-variables}

\href{https://ocw.mit.edu/courses/mathematics/18-03sc-differential-equations-fall-2011/unit-i-first-order-differential-equations/basic-de-and-separable-equations/MIT18_03SCF11_s1_2text.pdf}{MIT OCW Reference}

\begin{gather*}
    \frac{dy}{dx}=x(y-1)\\
    \frac{dy}{y-1}=xdx\\
    \ln{|y-1|}+C=\frac{x^2}{2}+C\\
    |y-1|=e^Ce^{\frac{x^2}{2}} \Leftrightarrow |y-1|=Ce^{\frac{x^2}{2}}\\
    y-1=\pm Ce^{\frac{x^2}{2}}\\
    y=1+Ce^{\frac{x^2}{2}}\\
\end{gather*}

\subsection{Exponential growth}\label{subsec:exponential-growth}

\[\frac{dP}{dt}=kP \Rightarrow P=P_0e^{kt}\]

A negative growth constant $k$ represents decay and a positive one represents growth.
For half life problems, one is solving for half (or any fraction or multiple) of the initial population.

\begin{gather*}
    \frac{1}{2}P_0=P_0e^{kt}\\
    \frac{1}{2}=e^{kt} \Rightarrow t=\frac{\ln{\frac{1}{2}}}{k}\\
\end{gather*}

\subsection{Euler's Method}\label{subsec:euler's-method}

When approximating a lower $x$-value than given, the procedure is still identical \textbf{but $\Delta{x}$ is negative}.

\[y_{new}=y_{old}+\frac{dy}{dx}|_{(x_{old},y_{old})}\Delta{x}\]

\subsection{Slope Fields}\label{subsec:slope-fields}

A differential equation in the form $\frac{dy}{dx}=F(x,y)$, where the slopes 
at each point in a 2D graph are plotted with line segments.
The solution curves for a differential equation align with these segments,
and the initial condition can be visualized.

\subsection{Logistic Growth}\label{subsec:logistic-growth}

Growth is limited by the carrying capacity $L$, which is found by setting the differential equation to 0
since this is when the curve flattens and the carrying capacity is achieved.
There is a positive growth constant $k$. $\lim_{t\to{\infty}}\frac{dP}{dt}=0$ always because 
of the asymptote to $L$.
However, $\frac{dP}{dt}$ \textbf{asymptotes to $0$} and is never exactly 0.

\[\frac{dP}{dt}=kP(L-P)\]

When derived, the population formula is as follows, with $A$ being a constant.

\[P(t)=\frac{L}{1+Ae^{-kt}}\]

The inflection point of the population (not growth) equation is when the population is growing the fastest.
This occurs at the time $t$ when $P=\frac{L}{2}$.
Furthermore, the carrying capacity of this population is given by
$\lim_{t\to{\infty}}P(t)=L$, which is always true due to form of $P(t)$.
