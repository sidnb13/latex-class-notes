\section{Quadratics and Polynomials}

\subsection{Basic factoring}

Given some $ax^2+bx+c$, try and find the factorization $(sx+u)(tx+v)$ by expanding this template expression.

\begin{remark}
    If one root is 0, the product of roots is 0 so
    the equation is of form $ax^2+bx=0$.
\end{remark}

\begin{remark}
    Difference of squares are of the form $x^2-a^2=0\implies (x-a)(x+a)=0$.
\end{remark}

\begin{remark}
    A perfect square is of the form $(x+a)^2=x^2+2ax+a^2$. This is a case of a "double root."
\end{remark}

\subsection{Quadratic Formula}

By manipulating $ax^2+bx+c=0$ by completing the square, the quadratic formula can be found.

\begin{equation}
    x=\frac{-b\pm \sqrt{b^2-4ac}}{2a}
\end{equation}

\subsection{Expansions}

\begin{eqnarray}
    (a+b)^2=a^2+2ab+b^2\\
    (a+b)^3 = a^3 + 3a^2b+3ab^2+b^3
\end{eqnarray}

\subsection{Factoring}

\begin{definition}[Difference of squares]
    \begin{equation}
        a^2-b^2=(a+b)(a-b)
    \end{equation}
\end{definition}

\begin{definition}[Sum of squares]
    \begin{equation}
        a^2+b^2=(a+b)^2-2ab
    \end{equation}
\end{definition}

\begin{definition}[Sum of cubes]
    \begin{equation}
        a^3+b^3=(a+b)(a^2-ab+b^2)
    \end{equation}
\end{definition}

\begin{definition}[Difference of cubes]
    \begin{equation}
        a^3-b^3=(a-b)(a^2+ab+b^2)
    \end{equation}
\end{definition}

\begin{definition}[Some cube identity]
    \begin{equation}
        a^3+b^3+c^3-3abc=(a+b+c)(a^2+b^2+c^2-ab-ac-bc)
    \end{equation}
\end{definition}

\begin{definition}[Simon factoring trick]
    Used to factor in a diophantine equation.
    If $ab + ka + nb = c$, then $(a+n)(b+k)=c+nk$.
\end{definition}

\subsection{Vieta's Formulas}

\begin{eqnarray}
    x^2+ax+b=(x-p)(x-q)\\
    x^2+ax+b=x^2-(p+q)x+pq
\end{eqnarray}

Thus, $a=p+q$ and $b=pq$. Generally,

\begin{align*}	 
    s_1&= & r_1+r_2+r_3&+\cdots+r_n & &=-\frac{a_{n-1}}{a_n} \\	 s_2&= & r_1r_2+r_1r_3+r_1r_4&+\cdots+r_{n-2}r_{n-1} & &=\phantom{-}\frac{a_{n-2}}{a_n} \\	 s_3&= & r_1r_2r_3+r_1r_2r_4&+\cdots+r_{n-2}r_{n-1}r_n & &=-\frac{a_{n-3}}{a_n} \\	 & & &\vdots & & \\	 s_n&= & r_1r_2r_3&\cdots r_n & &=(-1)^n\frac{a_0}{a_n}.\\	 
\end{align*}

\begin{definition}[AM-GM Inequality]
    \begin{equation}
        \frac{a_1+\ldots+a_n}{n}\geq \sqrt[n]{a_1a_2\ldots a_n}
    \end{equation}
\end{definition}