\section{Functions and Polynomials}

\subsection{Floor function}

The floor function $\lfloor x \rfloor$ yields greatest integer
leq to argument. For positive values, equivalent to rounding down (truncating decimals).
For negative values, equivalent to next lowest negative integer.

\begin{example}
    $$\lfloor -3.2 \rfloor = -4$$
\end{example}

A useful simplification is

\begin{equation}
    \lfloor x \rfloor = \lfloor y+k \rfloor
\end{equation}

where $y$ is an integer and $0 \leq k < 1$. An alternate definition
is

\begin{equation}
    \lfloor x \rfloor = x - \{x\}
\end{equation}

where $\{x\}$ is the fractional component of $x$.

\subsection{Change of base formula}

Express a logarithm in base $b$ using logarithms of base $d$.
Let $d,a,b\in \R$ s.t $d,b\neq 1$.

\begin{equation}
    \log _{b} a=\frac{\log _{d} a}{\log _{d} b}
\end{equation}

\subsection{Polynomial division}

\begin{definition}[Polynomial remainder theorem]
    Upon dividing any polynomial $P(x)$ by linear polynomial $x-a$, the remainder is $P(a)$.
\end{definition} 

We can express $P(x)$ as the following

\begin{equation}
    P(x)=(x-a)Q(x)+R(x)
\end{equation}

where $x-a$ is the dividend, $Q(x)$ is the quotient and $R(x)$ is the remainder.
Also, $\operatorname{deg} R(x)<\operatorname{deg}(x-a)$, hence $R(x)\in \R$.

The general approach is $P(x)=D(x)Q(x)-R(x)\implies R(x)=D(x)Q(x)-P(x)$.
Find zeros of $D(x)$ to eliminate $Q(x)$ and thus find $R(x)$ through substitution into $-P(x)$.

\subsection{Inverse functions}

\begin{definition}[Inverse of a function]
    Let $f(x):A\to B$ with range $C$. The inverse is $f^{-1}(x):C\to A$ iff $f$ is injective (i.e. a distinct one-to-one mapping from every value in $A$ to $C$).
    Can verify via horizontal line test.
\end{definition}

The properties are $f(f^{-1}(x))=x$ and $f^{-1}(f(x))=x$