\section{Planes}

\begin{center}
    \includegraphics[]{normal-plane}
\end{center}

If $(x,y,z)$ is a point on the plane, then given $\vec{P}_0=(x_0,y_0,z_0)$, $(x-x_0,y-y_0,z-z_0)$ is a vector on the plane perpendicular to $\vec{n}$, the
normal vector. Thus, $(A,B,C)\cdot(x-x_0,y-y_0,z-z_0)=0$ where $A,B,C$ are vector coordinates of $\vec{n}$. With expansion:

\begin{align*}
    A(x-x_0)+B(y-y_0)+C(z-z_0)&=0\\
    Ax+By+Cz=Ax_0+By_0+Cz_0&=0\\
    Ax+By+Cz=\vec{n}\cdot \vec{P}_0
\end{align*}

Note that $(A,B,C)$ form coordinates of $\vec{n}$.

To find a plane containing 3 points $\vec{v}_1,\vec{v}_2,\vec{v}_3$, compute, for example $\vec{c}_1=\vec{v}_3-\vec{v}_1$ and $\vec{c}_2=\vec{v}_2-\vec{v}_1$.
This finds 2 vectors in the plane. Then compute $\vec{c}_1\times \vec{c}_2=\vec{n}$. \newline

\noindent
The trace of a plane is the intersection of a plane $\mathcal{P}$ with $xy$, $xz$, or $yz$ coordinate planes. Can be found by setting respective variable to 0.

\subsection{Cross-product rules and identities}

\begin{center}
    \includegraphics[]{cross-product}
\end{center}

Overview
\begin{itemize}
    \item $||\vec{a}\times \vec{b}||=||\vec{a}||||\vec{b}||\sin\theta$
    \item $\vec{a},\vec{b}\perp \vec{a}\times \vec{b}$
\end{itemize}

Algebraic
\begin{itemize}
    \item $\vec{a}\times \vec{b}=\vec{0}$
    \item $\vec{a}\times \vec{b}=-\vec{b}\times \vec{a}$
    \item Distributive properties hold -- preserve direction however
    \item $(\alpha \vec{a})\times \vec{b}=\alpha(\vec{a}\times \vec{b})$
\end{itemize}

\section{Graphs}

\subsection{Multivariable functions}

Function of $n$-variables is real-valued function with $f(x_1,\cdots,x_n)$ with domain $\mathcal{D}$
being a set of $n$-tuples $(x_1,\cdots,x_n)$ in $\R^n$, or where $f$ is defined.
Range of $f$ is all values $f(x_1,\cdots,x_n)$ for $(x_1,\cdots,x_n)$ in the domain.

\subsection{Graphing multivariable functions}

Traces are 2D curves obtained by intersection with planes parallel to coordinate plane.
\begin{itemize}
    \item Horizontal trace at height $c$ -- intersection of graph with plane $z=c$, so points $(x,y,c)$ such that $f(x,y)=c$
    \item Vertical trace in plane $x=a$ -- intersection of graph with vertical plane $x=a$ for all points $(a,y,f(a,y))$
    \item Vertical trace in plane $y=b$ -- intersection of graph with vertical plane $y=b$ for all points $(x,b,f(x,b))$
\end{itemize}

\begin{center}
    \includegraphics[scale=0.15]{saddle-plot.png}
\end{center}

Saddle surface general form is $f(x,y)=x^2-y^2$. The horizontal traces are hyperbolas of the form $c=x^2-y^2$.
Vertical traces are parabolas, as either $x,y$ set to 0.\newline

\noindent
Linear functions in 2 variables are of the form $f(x,y)=mx+ny+r|m,n,r\in \R$.

\subsection{Contour maps and level curves}

\begin{center}
    \includegraphics[scale=0.15]{contour-map.png}
\end{center}

Can specify a contour interval for each $z=c$ value. Is a 2D representation of
level curves of $f(x,y)$ at an interval. Going along level curve means change in altitude is
0. Altitude has change of $\pm m$ (contour interval) when going up/down contour levels. Average ROC is $\Delta \text{elevation}/\Delta \text{distance}$.
Path of steepest ascent follows the shortest possible segment from one contour line to another and always points in steepest direction.

\begin{center}
    \includegraphics[scale=0.3]{steepest-ascent.png}
\end{center}

\section{Partial Derivatives}

\subsection{Definition}

If $f:\R^2\rightarrow \R$ is given by $f(x,y)=z$ abd $P_0=(a,b)$ is a point in the domain of $f$,
then the partial derivative are:
\begin{itemize}
    \item If $h:\R\rightarrow \R$ by $h(t)=f(t,b)$, then partial derivative with respect to $x$ at $P_0$
    is $h\,'(a)$ with following limit definition

    \[\left.\frac{\partial f}{\partial x}\right|_{(a, b)}=f_{x}(a, b)=\lim _{h \rightarrow 0} \frac{f(x+h, b)-f(x, b)}{h}=\lim _{x \rightarrow a} \frac{f(x, b)-f(a, b)}{x-a}\]

    \item If $g:\R\rightarrow \R$ by $g(t)=g(t,b)$ then partial derivative with respect to $y$ at $P_0$ is $g\,'(b)$ with following limit definition
    
    \[\left.\frac{\partial f}{\partial y}\right|_{(a, b)}=f_{y}(a, b)=\lim _{h \rightarrow 0} \frac{f(a, x+h)-f(a, b)}{h}=\lim _{y \rightarrow b} \frac{f(a, y)-f(a, b)}{y-b}\]

\end{itemize}

Can be thought of as the intersection of the plane shifted by $b$ with $f$, and the derivative of the resulting trace.

\subsection{Linear approximation with planes}

Let $z=f(x,y)$ be a scalar-valued function in $\R^2$ and $P_0=(a,b)$ be a point in domain of $f$. Can have 2 slope vectors representing partial derivatives:
$(1,0,f_x(a,b))$ and $(0,1,f_y(a,b))$. Can find a linear approximation by finding set of points in plane spanned by these vectors passing through
$(a,b,f(a,b))$.

\[\vec{n}=(1,0,f_x(a,b))\times(0,1,f_y(a,b))=(-f_x(a,b),-f_y(a,b),1)\]

Building the plane:

\begin{align*}
    (x-a, y-b, z-f(a, b)) \cdot \vec{n} &=0 \\
    (x-a, y-b, z-f(a, b)) \cdot\left(-f_{x}(a, b),-f_{y}(a, b), 1\right) &=0 \\
    -f_{x}(a, b)(x-a)-f_{y}(a, b)(y-b)+z-f(a, b) &=0
\end{align*}

Thus,

\[z=f(a, b)+f_{x}(a, b)(x-a)+f_{y}(a, b)(y-b)\]

\subsection{Higher-order derivatives}

Can be calculated using derivatives of $f_x$ and $f_y$. Notation:

\[f_{xx}=\frac{\partial}{\partial x}(\frac{\partial f}{\partial x}),\; f_{yy}=\frac{\partial}{\partial y}(\frac{\partial f}{\partial y})\]

Can also have mixed partials (read as with respect to $x$ or $y$):

\[f_{xy}=\frac{\partial}{\partial y}(\frac{\partial f}{\partial x}),\; f_{yx}=\frac{\partial}{\partial x}(\frac{\partial f}{\partial y})\]

By Clairaut's Theorem, if $f_{xy}$ and $f_{yx}$ are both continuous functions on a disk $D$, then $f_{xy}(a,b)=f_{yx}(a,b)\: \forall (a,b)\in D$. Means that $f_{xyxy}=f_{xxyy}=f_{yyxx}=f_{yxyx}$.

\section{Extrema}

\subsection{Definition and proofs}

A function $f$ has:
\begin{itemize}
    \item local maximum at $P_0$ in domain if $f(P_0)>f(x,y)\;\forall\;(x,y)$ sufficiently near $P_0$
    \item local minimum at $P_0$ in domain if $f(P_0)<f(x,y)\;\forall\;(x,y)$ sufficiently near $P_0$
\end{itemize}

Sufficiently near: positive radius $R$ used to build a circle centered at $P_0$ that traps points in domain with desired property (min or max).
Global extrema redefine sufficiently near as in the domain of $f$.

\par 
Critical point is defined as either of the following:
\begin{itemize}
    \item $f_x(P_0)=f_y(P_0)=0$
    \item either $f_x(P),f_y(P)$ does not exist
\end{itemize}

Method: find critical value $y$ from $f_y$ and $x$ from $f_x$ through cross-substitution.

\par
Proof that if $f_x(P),f_y(P)$ both exist and there is a local max at $P$, then both partials are 0:

\par
Define $g:\;\R\rightarrow \R$ to be single variable function from holding $y=b$ in $f$. Considering points sufficiently near:

\includegraphics[scale=0.5]{Screen Shot 2021-03-21 at 7.55.34 PM.png}

Plugging into function $g$: $g(x) < f(x,b) \leq f(P_0)=g(a)$, so for $x$-values sufficiently
near $a$, $g(x)\leq g(a)$, so it is a local max of $g$. Invoking theorem of extrema, either $g'(a)=0$ or DNE. As $g'(a)=f_x(P_0)$, demonstrates that
$f_x=0$ or DNE for a local max. Same can be done for $f_y$.

\par
Can minimize a function's distance to origin through distance formula. As a square root minimizes
at its argument, can just concentrate on minimizing the argument.

\par
Reiteration: If $f: \mathbb{R}^{2} \rightarrow \mathbb{R}$ has continuous first and second order partial derivatives then $f_{x y}(x, y)=f_{y x}(x, y)$.
Continuous 2nd order partials: $C^2$; continous $n$ order partials: $C^\infty$.

\subsection{Second-derivative test}

Given a function $f:\R^2 \rightarrow \R$ that has continuous 2nd partials near a critical point $P$, define
the discriminant of $f$ at $P$ to be:

\[D(P)=f_{x x}(P) f_{y y}(P)-\left(f_{x y}(P)\right)^{2}\]

Then:
\begin{itemize}
    \item If $D>0$, $P$ is a local extreme of $f$
    \begin{itemize}
        \item $f_{xx}(P)>0$ implies a local min
        \item $f_{xx}(P)<0$ implies a local max
    \end{itemize}
    \item If $D<0$ then there is neither a min or max at $P$, but an inflection point (saddle point)
    \item If $D=0$ then there is no information about $P$
\end{itemize}

Observe that $D(P)=\operatorname{det}\left(\begin{array}{ll}
                f_{x x}(P) & f_{x y}(P) \\
                f_{y x}(P) & f_{y y}(P)
            \end{array}\right)$

\subsection{Global extrema}

If $f:\R^2 \rightarrow \R$ is continuous on a closed and bounded subset of $\R^2$ then it has a global max/min
on the subset (at critical point or along boundary)

\par
A closed subset of $\R^2$ is one that contains the boundary. Boundary points have the property that
any circle centered around them with positive radius will contain points in and out of the subset.
Bounded subset is where any distance between 2 points in the set never exceeds some fixed bound $M\in \R$.

\begin{center}
    \includegraphics[scale=0.5]{Screen Shot 2021-03-22 at 3.01.05 PM.png}
\end{center}

The boundary curve is denoted as $\partial D$ where $D$ is a disk. By parameterizing $\partial D$ and using function
composition to take this curve $\vec{c}(t)$ into $f$: $h(t)=f(\vec{c}(t))$. A min/max for $f$ along boundary curve
is a min/max of $h$. Plug resulting coordinates into $f$ and determine global extrema.
