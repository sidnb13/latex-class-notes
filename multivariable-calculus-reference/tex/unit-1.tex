\section{Paths whose image curve is a circle}

\subsection{Unit Circle}

Unit circle is set of points in $\R^2$ defined as $C=\{(x,y)\in\mathbb{R} ^2|x^2+y^2=1\}$. Ellipse is
$C=\{(x,y)\in\mathbb{R} ^2|\frac{x^2}{a^2}+\frac{y^2}{b^2}=1\}$
Has standard parameterization of $\vec{c}(t)=\big(\cos(t),\sin(t)\big)$.
When parameterizing, always start from $t=0$ reference unless otherwise given.\newline 

\begin{center}
    \includegraphics[scale=0.3]{unit-circle.png}
\end{center}

\noindent Properties
\begin{itemize}
    \item Image of $\vec{c}$ is a closed curve (has no endpoints, plane is divided into $\geq 2$ disjoint regions)
    \item Image of $\vec{c}$ is a simple curve; no self-intersection 
    \item $\vec{c}(t)$ is an \textbf{injective path}; path is considered injective if $\vec{c}(t_1)=\vec{c}(t_2)$, which implies that $t_1=t_2$ where 
    these are on the open interval $(a,b)$ even if $a=b$
    \item Orientation of $\vec{c}$ is counter-clockwise in traversal
\end{itemize}

\subsection{Observations}

\[\boxed{\vec{p}(t)=(a\cos (\pm (nt \pm\theta)) + x_0, b\sin (\pm (nt \pm \theta)) + y_0)}\]

If $-t$ for $t$, orientation is CW, CCW is $t$. If $a=b$, then curve is 
a circle of radius $a$ or $b$, else an ellipse with horizontal and vertical radii.
$x_0$ and $y_0$ simply shift the center coordinate. $n>0\in \R$ determines how many times
the circle is traversed given $t\in[0,2\pi]$, for example. $\theta$ is the phase shift.
When changing direction of traversal, cannot have $a>b$ for $[a,b]$ so to decrease argument of $\sin$ or $\cos$
must have $-t$ for $t$. Starting out, $-t$ goes through the angle range and $t$ is just a sign flip.

\section{Paths whose image is a line or line segment in the plane}

\subsection{Line Parametrics}

A line is a 1D subspace of $\R^2$, so $L=\{t\vec{m}|t\in\R\}$ for $\vec{m}\in\R^2$.
$\vec{m}=\begin{bmatrix}m_x\\m_y\end{bmatrix}$ is the \textbf{slope vector}.
Path given by image of $L$: 

\[\vec{c}(t)=\left(m_{x} t, m_{y} t\right),t\in\R\]

Can represent $\vec{c}(t)=t\vec{m}$ as well.\newline

\noindent
Lines Main Ideas
\begin{itemize}
    \item Image of a line is a curve (e.g. $y=x$ represents image curve of $\vec{c}(t)=(t,t)$)
    \item Lines can have nonzero intercepts, so $\vec{c}(t)=t\vec{m}$ represents $y=2x+1$. Line
    that has intercept vector $P_0=(x_0,y_0)$ $\parallel$ $\vec{m}=(m_x,m_y)$ can be expressed as:

    \[\boxed{\vec{c}(t)=(x_0+tm_x, y_0+m_yt)=\vec{P_0}+t\vec{m}}\]

    Note endpoint of $\vec{c}(t)$ is on image line (curve).

\end{itemize}

\subsection{General Forms}

2 parametric lines \textbf{collide} if they intersect and the point of intersection corresponds
to the same $t$ in both curves. If you set the parameter vector coordinates equal to each
other and solve for $t$, a solution indicates they collide. Intersection is found by \textbf{eliminating}
the parameter (solve for $t$ in terms of either $x$ or $y$ and plug into the other).\newline

\noindent
General form of parameterized curve can be expressed as the following:

\[\boxed{\vec{c}(t)=(\frac{m_x}{\Delta t}(t-a)+x_0,\frac{m_y}{\Delta t}(t-a)+y_0)}\]

where $\Delta t$ is the domain interval over $[a,b]$ and $(x_0,y_0)$ represents the desired \textbf{starting coordinate}.
This is important as when going in reverse, other coordinate can be used and slope might be negative.
$a$ is used in $(t-a)$ because everything is conventionally done with respect to starting coordinate.

\section{Paths whose image curve is a line in R3}

\subsection{R3 parameterization}

If $\vec{m}$ is a nonzero vector along $L$ through origin in $\R^3$, then $L=\{t\vec{m}|t\in\R\}$;
follows that $\vec{m}=(m_x,m_y,m_z)$, the slope or direction vector of the line. The basic parameterization is:

\[\vec{c}(t)=(m_x t,m_y t,m_z t)\]

Basis vectors in $\R^3$ are $\vec{i}, \vec{j}, \vec{k}$. Rewriting parameterization: 

\[\vec{c}(t)=(x_0+m_x t) \vec{i}+(y_0+m_y t) \vec{j}+(z_0+m_z t) \vec{k}\]

2 lines $\vec{c}_1(t)=P_0+\vec{m}_1t$ and $\vec{c}_2(t)=Q_0+\vec{m}_2t$ are parallel if
direction vectors are parallel ($\vec{m_1}=k\vec{m}_2$). Collisions still exist. If neither parallel nor intersecting, considered as skew.\newline

\noindent
To determine skew, parallel, or coincide, use parameters $s,t$ for each line and solve SOE.
If same slope, rule out skew clearly, then check if $s,t\in\R$: if not, then parallel, if so, then they coincide. If intersecting and want to check if collide, some $t$
must satisfy all relations.

\section{Cycloid Problem}

\begin{center}
    \includegraphics*[scale=0.9]{cycloid.png}
\end{center}

With radius 1 and passing through the origin:

\[\vec{c}(t)=(t-\sin t, 1 - \cos t)\]

Observe that:

\[\vec{c}^\prime(t)=(1-\cos t, \sin t)\]

Can define the vector $\vec{u}=\left(\begin{array}{c}x^{\prime}(t) \\0\end{array}\right)$
such that $\vec{u}$ is always horizontal and $||\vec{u}||=|x^\prime(t)|$. Reaches maximum
value at $t\in[k\pi|k\in\R]$ and is has minimum cusp where it is 0 at $t\in [2k\pi|k\in\R]$.
Thus, $x^\prime(t)\geq 0$ always, as the x-coordinate is never decreasing. \newline

\noindent
Can also define the vector $\vec{v}=\left(\begin{array}{c}0 \\y^{\prime}(t)\end{array}\right)$
with the same properties. Reaches maximum value when $t\in[k\frac{\pi}{2}|k\in\R]$.
Can change, as observe $t$ when $\sin t < 0$ or $ >0 $.

\subsection{Hypercycloid Derivation}

\begin{center}
    \includegraphics*[scale=0.5]{Screen Shot 2021-02-06 at 9.40.41 AM.png}
\end{center}

\subsection{Hypocycloid Derivation}

\begin{center}
    \includegraphics*[scale=0.6]{Screen Shot 2021-02-06 at 9.58.04 AM.png}
\end{center}

\section{Velocity Vector}

\subsection{Definitions}

Vector $\vec{u}(t_0)+\vec{v}(t_0)$ is he velocity vector to the curve $\vec{c}(t)$ at $t=t_0$.\newline

\noindent
Let $\vec{c}:\;[a,b]\rightarrow \R^n$ have a path $\vec{c}(t)=\left(x_{1}(t), x_{2}(t), x_{3}(t), \ldots, x_{n}(t)\right)$ (let $x_i(t):[a,b]\rightarrow \R$ for each $i$)
\begin{itemize}
    \item If $t_0\in[a,b]$, then $\vec{c}^{\prime}\left(t_{0}\right):=\left(x_{1}^{\prime}\left(t_{0}\right), x_{2}^{\prime}\left(t_{0}\right), x_{3}^{\prime}\left(t_{0}\right), \ldots, x_{n}^{\prime}\left(t_{0}\right)\right)$; the velocity vector to $\vec{c}$ at $t_0$
    \item The path $\vec{c}^{\prime}\left(t_{0}\right):=\left(x_{1}^{\prime}\left(t_{0}\right), x_{2}^{\prime}\left(t_{0}\right), x_{3}^{\prime}\left(t_{0}\right), \ldots, x_{n}^{\prime}\left(t_{0}\right)\right)$; the velocity vector to $\vec{c}$ is referred to as velocity of $\vec{c}(t)$
\end{itemize}

Recall chain rule: if $y=f(x)$ where $x$ is a function of $t$, $y'(t)=x'f'(x)$, not to be confused with product rule.
Can write $f'(x)=\frac{y'(t)}{x'(t)}$

\begin{itemize}
    \item If $\vec{p}(t)=\vec{c}(t)+\vec{r}(t),$ then $\vec{p}^{\prime}(t)=\vec{c}^{\prime}(t)+\vec{r}^{\prime}(t)$
    \item If $g(t)=\vec{c}(t) \cdot \vec{r}(t),$ then $g^{\prime}(t)=\vec{c}^{\prime}(t) \cdot \vec{r}(t)+\vec{c}(t) \cdot \vec{r}^{\prime}(t)$
    \item If $\vec{p}(t)=f(t) \vec{c}(t),$ then $\vec{p}^{\prime}(t)=f^{\prime}(t) \vec{c}(t)+f(t) \vec{c}^{\prime}(t)$
    \item If $\vec{p}(t)=\vec{c}(t) \times \vec{r}(t),$ then $\vec{p}^{\prime}(t)=\vec{c}^{\prime}(t) \times \vec{r}(t)+\vec{c}(t) \times \vec{r}^{\prime}(t)$
    \item If $\vec{p}(t)=\vec{c}(f(t)),$ then $\vec{p}^{\prime}(t)=f^{\prime}(t) \vec{c}^{\prime}(f(t))$
    \item If $g(t)=\|\vec{c}(t)\|,$ then $g^{\prime}(t)=\frac{\vec{c}(t) \cdot \vec{c}^{\prime}(t)}{\|\vec{c}(t)\|}$
\end{itemize}

\subsection{Tangent Line}

Tangent line can be visualized as a base vector in standard position plus a velocity vector
tangent to the tip which traces a shifted line in some interval. General formula with base vector $\vec{c}(t_0)$ and slope $\vec{c}'(t_0)$:

\[\boxed{\ell(t)=\vec{c}(t_0)+(t-t_0)\vec{c}'(t_0)}\]

\section{Space Curves}

\begin{itemize}
    \item Projection into the $x y$ plane is the path $(x(t), y(t), 0)$.
    \item Projection into the $x z-$ plane is the path $(x(t), 0, z(t))$.
    \item Projection into the $y z$ plane is the path $(0, y(t), z(t))$.
\end{itemize}

\section{Speed and Arclength}

\subsection{Speed}

Speed of a parametric function in $\R^n$ is given by:

$$||\vec{c}'(t)||=\sqrt{\displaystyle\sum_{i=1}^{n}c_i(t)^2}$$

(being the magnitude of the velocity vector)

\subsection{Arclength}

Arclength of a parametric function is given by:

\[S=\int_a^b ||\vec{c}'(t)||dt=\int_a^b\sqrt{(\frac{dx}{dt})^2+(\frac{dy}{dt})^2+(\frac{dz}{dt})^2+\cdots}\;dt\]

Can approximate arclength as a sum of the lengths of secant vector approximations
$\vec{s}_i=\vec{c}(t_i)-\vec{c}(t_{i-1})$:

\[\mbox{arclength}\approx\sum_{i=1}^n||\vec{s}_i||\]

According to the MVT, there exists a $\hat{t}_i$ in $(t_{i-1},t_i)$ (open interval due to differentiability requirement) such that:

\begin{align*}
    x'(\hat{t}_i)&=\frac{x(t_i)-x(t_{i-1})}{t_i-t_{i-1}}\\
    y'(\hat{t}_i)&=\frac{y(t_i)-y(t_{i-1})}{t_i-t_{i-1}}
\end{align*}

This means that, since $\vec{s}_i$ is given as the difference between 2 points,
being a secant:

\begin{align*}
    \vec{s}_i&=\left((t_i-t_{i-1})x(\hat{t}_i), (t_i-t_{i-1})y(\hat{t}_i)\right )\\
    \vec{s}_i&=(t_i-t_{i-1})\left(x'(\hat{t}_i),y'(\hat{t}_i)\right)\\
    \vec{s}_i&=(t_i-t_{i-1})\vec{c}\,'(\hat{t}_i)
\end{align*}

Thus,

\begin{align*}
    \mbox{arclength}&\approx\sum_{i=1}^n||\vec{s}_i||\\
    \mbox{arclength}&\approx\sum_{i=1}^n||\Delta t \,\vec{c}\,'(\hat{t}_i)||\\
    \mbox{arclength}&\approx\sum_{i=1}^n\Delta t||\vec{c}\,'(\hat{t}_i)||
\end{align*}

Can define the arclength differential as follows:

\[\mathrm{d}s=\sqrt{\mathrm{d}x^2+\mathrm{d}y^2}\]

Can just define arclength as $\text{arclength}=\int \mathrm{ds}$

\subsection{Arclength Parameterization}

Higher the speed of a curve, farther the points are spaced apart.
An arclength parametrization of a curve is a path whose image is the desired curve and whose speed is constantly one.
Or, $\vec{c}:[a,b]\to\mathbb{R}^n$ with $||\vec{c}\,'(t)||=1$ for $t\in[a,b]$.
If a curve is not an arclength parameterization, then can do $\frac{\vec{c}(t)}{||\vec{c}'(t)||}$ but only dividing the coefficients (slopes).\newline

\noindent
When speed is variable, is difficult to define arclength parameterization.
Thus, can define displacement to be $s(t)=\int_a^b (t)dt$. If $v(t)!\neq 0$,
then $s$ is injective because according to FTC, $s'(t)=v(t)$.
By definition, $v'(t)\geq 0$ always since it is composed of a radical, so it must be \textbf{increasing}.
Thus, if $t_1=t_2$, $s(t_1)\neq s(t_2)$. Arclength parameterization:

\[s(t)=\int_0^t||\vec{c}'(u)||du\]

\noindent
This means that $s$ is invertible, so can solve for $t$ to get $t=\varphi(s)$.
An arclength parameterization can be found by:

\[\boxed{\vec{p}(s)=\vec{c}(\varphi(s))}\]

\section{Curvature}

\subsection{Proofs}

Recall that to make an arclength parameterization accumulate the magnitudes of infinitesimal velocity vectors:

\[s(t)=\int_a^t ||\vec{r}'(u)||du\]

Given some curve $\vec{r}(t)$, define an arclength parameterization by $\vec{r}(g(s))\rightarrow \vec{r}_1(s)$, so 
$\vec{r}$ is defined in terms of $s$. The unit tangent vector $\vec{T}_1(s)$ is then $\frac{\vec{r}_1'(s)}{||\vec{r}_1'(s)||}=\vec{r}_1'(s)$.

\begin{align*}
    \vec{T}_1(s)&=\vec{r}_1'(s)\\
    &=\frac{d}{ds}\vec{r}_1(s)\\
    &=\frac{d}{ds}\vec{r}(g(s))\\
    &=\vec{r}'(g(s))\cdot g'(s)=\vec{r}'(t)\cdot \frac{dt}{ds}\\
    &=\frac{\vec{r}'(t)}{\frac{ds}{dt}}=\frac{\vec{r}'(t)}{||\vec{r}'(t)||}
\end{align*}

This means that $\vec{T}_1(s)=T(t)$

Continuing, to find curvature $\kappa(t)$:

\begin{align*}
    \vec{T}_1'(s)&=\frac{d}{ds}\vec{T}(t)\\
    &=\frac{d}{ds}\vec{T}(g(s))\\
    &=\vec{T}'(t)\cdot \frac{dt}{ds}\\
    &=\frac{\vec{T}'(t)}{\frac{ds}{dt}}\\
    &=\frac{\vec{T}'(t)}{||\vec{r}'(t)||}\\
\end{align*}

Thus, $\boxed{\kappa(t)=\frac{\vec{T}'(t)||}{||\vec{r}'(t)||}}$

\subsection{Definition}

Given a curve $C$ parameterized with arclength by the path $\vec{c}:[a,b]\rightarrow \R^n$,
curvature is defined as:

\[\boxed{\kappa(s)=||\vec{T}'(s)||}\]

where $\vec{c}'(s)\neq 0$ and $\vec{T}(s)=\frac{\vec{c}'(s)}{||\vec{c}'(s)||}$ (normalized slope vector).\newline

A loose geometric interpretation is that a greater $\kappa(s)$ implies
more curvature, that is, the curve is changing at a greater rate there.
When $\vec{c}'(s)\neq 0$ is always true for a curve, it is \textbf{regular}.
Is defined in terms of arclength parameterization so curvature
is an intrinsic property of the curve independent of parameterization.\newline

\noindent
Formula for curvature at the point $\vec{c}(t)$:

\[\boxed{\kappa(t)=\frac{||T'(t)||}{||\vec{c}\,'(t)||}=\frac{||\vec{c}\,'(t)\times \vec{c}\,''(t)||}{||\vec{c}\,'(t)||^3}}\]

\section{Motion in 3D space}

Given a path $\vec{c}: \mathbb{R} \rightarrow \mathbb{R}^{3}$ with $\vec{c}(t)=(x(t), y(t), z(t)),$ then we have defined:
\begin{itemize}
    \item $\cdot \vec{v}(t)=\vec{c}^{\prime}(t)=\left(x^{\prime}(t), y^{\prime}(t), z^{\prime}(t)\right)$ is also path in $\mathbb{R}^{3}$ called the velocity of $\vec{c}$
    \item $\vec{a}(t)=\vec{v}^{\prime}(t)=\vec{c}^{\prime \prime}(t)=\left(x^{\prime \prime}(t), y^{\prime \prime}(t), z^{\prime \prime}(t)\right)$ is also a path in $\mathbb{R}^{3}$ called the acceleration of $\vec{c}$
    \item $v(t)=\|\vec{v}(t)\|=\left\|\vec{c}^{\prime}(t)\right\|$ is a scalar valued function on $\mathbb{R}$ (that's a fancy way of saying the domain and codomain of this function are both $\mathbb{R}$ ) called the speed of $\vec{c}$
    \item $\vec{T}(t)=\frac{\vec{c}^{\prime}(t)}{v(t)}$ is also a path in $\mathbb{R}^{3}$ called the unit tangent to $\vec{c}$
    \item $\kappa(t)=\frac{\vec{T}^{\prime}(t)}{v(t)}=\frac{\left\|\vec{c}^{\prime}(t) \times \vec{c}^{\prime \prime}(t)\right\|}{\left\|\vec{c}^{\prime}(t)\right\|^{3}}$ is a scalar valued function on $\mathbb{R}$ called the curvature of $\vec{c}$
\end{itemize}

Note that $\vec{T}\cdot \vec{T}=||\vec{v}||^2=1$.
Computing the derivative, $\frac{\mathrm{d}}{\mathrm{d} t} \vec{T} \cdot \vec{T}=2 \vec{T} \cdot \vec{T}^{\prime}=0$
This means that $\vec{T}\perp \vec{T}'$.\newline

\noindent
Define $\boxed{\vec{N}(t)=\frac{\vec{T}^{\prime}(t)}{\left\|\vec{T}^{\prime}(t)\right\|}}$ as
the unit normal vector, which is the unit tangent to the unit tangent. From observation, $\vec{T}\perp \vec{N}$.
Fact: the acceleration vector always lies in the plane spanned by $\vec{N}$ and $\vec{T}$.\newline

\noindent
Acceleration $\vec{a}(t)$ is thus split component-wise into $a_T$ from $\vec{T}$
and $a_N$ from $\vec{N}$:

\[\boxed{a_T=v'(t)=\frac{\vec{a(t)}\cdot \vec{v}(t)}{v(t)}}\]
\[\boxed{a_N=\kappa(t)v(t)^2=\frac{||\vec{a}(t)\times \vec{v}(t)||}{v(t)}=\sqrt{\|\vec{a}(t)\|^{2}-\left|a_{T}\right|^{2}}}\]

\section{Derivatives of parameterized curves}

\subsection{Arclength parameterization derivation}

Take the following function:

\[\vec{r}(t)=\langle x(t),y(t) \rangle\]

An arclength parameterization is achieved with the following computation:

\[s=\int_0^t \sqrt{x'(u)^2+y'(u)^2}\;du\]

Can say that $t=g(s)$, so the arclength parameterization, which is the path in terms of $s$:

\[\vec{r}_1(s)=\langle x(g(s)),y(g(s))\rangle\]

Taking the derivative by the chain rule:

\begin{align*}
    \vec{r}_1'(s)&=\langle x'(g(s))\cdot g'(s),y'(g(s))\cdot g'(s)\rangle \\
    &=g'(s)\langle x'(t),y'(t) \rangle \\
\end{align*}

Note that $g'(s)=\frac{dt}{ds}=\frac{1}{\frac{ds}{dt}}=\frac{1}{||\vec{r}'(t)||}$ by taking the derivative of the integral for arclength:

\[\vec{r}_1'(s)=\frac{1}{||\vec{r}'(t)||}\langle x'(t),y'(t) \rangle=\vec{T}(t)\]

Following from this, $g'(s)=\frac{1}{||\vec{r}'(g(s))||}$ so $g''(s)=-\frac{1}{||\vec{r}'(g(s))||^2}\cdot g'(s)=-\frac{1}{||\vec{r}'(t)||^3}$

\subsection{Orthogonal derivative and position vectors}

Observe that $\vec{r}\cdot \vec{r}=||\vec{r}||^2$.
Thus, $\frac{d}{dt}[\vec{r}\cdot \vec{r}]=2\vec{r}\cdot\vec{r}'=2||\vec{r}||||\vec{r}||'$.
Rearranging: $\frac{\vec{r}\cdot \vec{r}}{||\vec{r}||}=||\vec{r}||'$.
Means that magnitude of position vector has to be a constant value in order for it
to be $\perp$ to derivative.

\section{Planetary motion}

\begin{itemize}
    \item Law of ellipses -- orbit of planet is ellipse with sun as focus
    \item Law of equal area in equal time -- position vector pointing from sun to planet sweeps out equal area in equal time (so speed must increase/decrease)
\end{itemize}

Can approximate the area swept in time by $\frac{dA}{dt}=\frac{1}{2}||\vec{r}(t)\times \vec{r}'(t)||=\frac{1}{2}||\vec{J}||$.
The differential equation for each of Kepler's laws is: $\vec{r}''(t)=-\frac{k}{||\vec{r}(t)||^3}\vec{r}(t)$, so it is in the direction of $\vec{r}(t)$.
Thus, differentiating $\frac{d\vec{J}}{dt}=\frac{d}{dt}(\vec{r}'(t)\times \vec{r}''(t))=0$.

\subsection{Cross-product identities}

Cross product identities:
\begin{itemize}
    \item $\boxed{\vec{u}\times(\vec{v}\times \vec{w})=(u\cdot w)\vec{v}-(\vec{u}\cdot \vec{v})\vec{w}}$
    \item $\boxed{u\cdot(\vec{v}\times \vec{w})=v\cdot(\vec{w}\times \vec{u})=\vec{w}\cdot(\vec{u}\times \vec{v})}$
\end{itemize}

\section{Planes}

If $(x,y,z)$ is a point on the plane, then given $\vec{P}_0=(x_0,y_0,z_0)$, $(x-x_0,y-y_0,z-z_0)$ is a vector on the plane perpendicular to $\vec{n}$, the
normal vector. Thus, $(A,B,C)\cdot(x-x_0,y-y_0,z-z_0)=0$ where $A,B,C$ are vector coordinates of $\vec{n}$. With expansion:

\begin{align*}
    A(x-x_0)+B(y-y_0)+C(z-z_0)&=0\\
    Ax+By+Cz=Ax_0+By_0+Cz_0&=0\\
    Ax+By+Cz=\vec{n}\cdot \vec{P}_0
\end{align*}

Note that $(A,B,C)$ form coordinates of $\vec{n}$.

To find a plane containing 3 points $\vec{v}_1,\vec{v}_2,\vec{v}_3$, compute, for example $\vec{c}_1=\vec{v}_3-\vec{v}_1$ and $\vec{c}_2=\vec{v}_2-\vec{v}_1$.
This finds 2 vectors in the plane. Then compute $\vec{c}_1\times \vec{c}_2=\vec{n}$. \newline

\noindent
The trace of a plane is the intersection of a plane $\mathcal{P}$ with $xy$, $xz$, or $yz$ coordinate planes. Can be found by setting respective variable to 0.
\subsection{Cross-product rules and identities}

Overview
\begin{itemize}
    \item $||\vec{a}\times \vec{b}||=||\vec{a}||||\vec{b}||\sin\theta$
    \item $\vec{a},\vec{b}\perp \vec{a}\times \vec{b}$
\end{itemize}

Algebraic
\begin{itemize}
    \item $\vec{a}\times \vec{b}=\vec{0}$
    \item $\vec{a}\times \vec{b}=-\vec{b}\times \vec{a}$
    \item Distributive properties hold -- preserve direction however
    \item $(\alpha \vec{a})\times \vec{b}=\alpha(\vec{a}\times \vec{b})$
\end{itemize}