\section{Paths whose image curve is a circle}

\subsection{Unit Circle}

Unit circle is set of points in $\R^2$ defined as $C=\{(x,y)\in\mathbb{R} ^2|x^2+y^2=1\}$. Ellipse is
$C=\{(x,y)\in\mathbb{R} ^2|\frac{x^2}{a^2}+\frac{y^2}{b^2}=1\}$
Has standard parameterization of $\vec{c}(t)=\big(\cos(t),\sin(t)\big)$.
When parameterizing, always start from $t=0$ reference unless otherwise given.\newline 

\begin{center}
    \includegraphics[scale=0.3]{unit-circle.png}
\end{center}

\noindent Properties
\begin{itemize}
    \item Image of $\vec{c}$ is a closed curve (has no endpoints, plane is divided into $\geq 2$ disjoint regions)
    \item Image of $\vec{c}$ is a simple curve; no self-intersection 
    \item $\vec{c}(t)$ is an \textbf{injective path}; path is considered injective if $\vec{c}(t_1)=\vec{c}(t_2)$, which implies that $t_1=t_2$ where 
    these are on the open interval $(a,b)$ even if $a=b$
    \item Orientation of $\vec{c}$ is counter-clockwise in traversal
\end{itemize}

\subsection{Observations}

\[\boxed{\vec{p}(t)=(a\cos (\pm n(t \pm\theta)) + x_0, b\sin (\pm n(t \pm \theta)) + y_0)}\]

If $-t$ for $t$, orientation is CW, CCW is $t$. If $a=b$, then curve is 
a circle of radius $a$ or $b$, else an ellipse with horizontal and vertical radii.
$x_0$ and $y_0$ simply shift the center coordinate. $n>0\in \R$ determines how many times
the circle is traversed given $t\in[0,2\pi]$, for example. $\theta$ is the phase shift.
When changing direction of traversal, cannot have $a>b$ for $[a,b]$ so to decrease argument of $\sin$ or $\cos$
must have $-t$ for $t$. Starting out, $-t$ goes through the angle range and $t$ is just a sign flip.

\section{Paths whose image is a line or line segment in the plane}

\subsection{Line Parametrics}

A line is a 1D subspace of $\R^2$, so $L=\{t\vec{m}|t\in\R\}$ for $\vec{m}\in\R^2$.
$\vec{m}=\begin{bmatrix}m_x\\m_y\end{bmatrix}$ is the \textbf{slope vector}.
Path given by image of $L$: 

\[\vec{c}(t)=\left(m_{x} t, m_{y} t\right),t\in\R\]

Can represent $\vec{c}(t)=t\vec{m}$ as well.\newline

\noindent
Lines Main Ideas
\begin{itemize}
    \item Image of a line is a curve (e.g. $y=x$ represents image curve of $\vec{c}(t)=(t,t)$)
    \item Lines can have nonzero intercepts, so $\vec{c}(t)=t\vec{m}$ represents $y=2x+1$. Line
    that has intercept vector $P_0=(x_0,y_0)$ $\parallel$ $\vec{m}=(m_x,m_y)$ can be expressed as:

    \[\boxed{\vec{c}(t)=(x_0+tm_x, y_0+m_yt)=\vec{P_0}+t\vec{m}}\]

    Note endpoint of $\vec{c}(t)$ is on image line (curve).

\end{itemize}

\subsection{General Forms}

2 parametric lines \textbf{collide} if they intersect and the point of intersection corresponds
to the same $t$ in both curves. If you set the parameter vector coordinates equal to each
other and solve for $t$, a solution indicates they collide. Intersection is found by \textbf{eliminating}
the parameter (solve for $t$ in terms of either $x$ or $y$ and plug into the other).\newline

\noindent
General form of parameterized curve can be expressed as the following:

\[\boxed{\vec{c}(t)=(\frac{m_x}{\Delta t}(t-a)+x_0,\frac{m_y}{\Delta t}(t-a)+y_0)}\]

where $\Delta t$ is the domain interval over $[a,b]$ and $(x_0,y_0)$ represents the desired \textbf{starting coordinate}.
This is important as when going in reverse, other coordinate can be used and slope might be negative.
$a$ is used in $(t-a)$ because everything is conventionally done with respect to starting coordinate.

\section{Paths whose image curve is a line in R3}

\subsection{R3 parameterization}

If $\vec{m}$ is a nonzero vector along $L$ through origin in $\R^3$, then $L=\{t\vec{m}|t\in\R\}$;
follows that $\vec{m}=(m_x,m_y,m_z)$, the slope or direction vector of the line. The basic parameterization is:

\[\vec{c}(t)=(m_x t,m_y t,m_z t)\]

Basis vectors in $\R^3$ are $\vec{i}, \vec{j}, \vec{k}$. Rewriting parameterization: 

\[\vec{c}(t)=(x_0+m_x t) \vec{i}+(y_0+m_y t) \vec{j}+(z_0+m_z t) \vec{k}\]

2 lines $\vec{c}_1(t)=P_0+\vec{m}_1t$ and $\vec{c}_2(t)=Q_0+\vec{m}_2t$ are parallel if
direction vectors are parallel ($\vec{m_1}=k\vec{m}_2$). Collisions still exist. If neither parallel nor intersecting, considered as skew.\newline

\noindent
To determine skew, parallel, or coincide, use parameters $s,t$ for each line and solve SOE.
If same slope, rule out skew clearly, then check if $s,t\in\R$: if not, then parallel, if so, then they coincide. If intersecting and want to check if collide, some $t$
must satisfy all relations.

\section{Cycloid Problem}

\begin{center}
    \includegraphics*[scale=0.9]{cycloid.png}
\end{center}

With radius 1 and passing through the origin:

\[\vec{c}(t)=(t-\sin t, 1 - \cos t)\]

Observe that:

\[\vec{c}^\prime(t)=(1-\cos t, \sin t)\]

Can define the vector $\vec{u}=\left(\begin{array}{c}x^{\prime}(t) \\0\end{array}\right)$
such that $\vec{u}$ is always horizontal and $||\vec{u}||=|x^\prime(t)|$. Reaches maximum
value at $t\in[k\pi|k\in\R]$ and is has minimum cusp where it is 0 at $t\in [2k\pi|k\in\R]$.
Thus, $x^\prime(t)\geq 0$ always, as the x-coordinate is never decreasing. \newline

\noindent
Can also define the vector $\vec{v}=\left(\begin{array}{c}0 \\y^{\prime}(t)\end{array}\right)$
with the same properties. Reaches maximum value when $t\in[k\frac{\pi}{2}|k\in\R]$.
Can change, as observe $t$ when $\sin t < 0$ or $ >0 $.

\section{Velocity Vector}

Vector $\vec{u}(t_0)+\vec{v}(t_0)$ is he velocity vector to the curve $\vec{c}(t)$ at $t=t_0$.\newline

\noindent
Let $\vec{c}:\;[a,b]\rightarrow \R^n$ have a path $\vec{c}(t)=\left(x_{1}(t), x_{2}(t), x_{3}(t), \ldots, x_{n}(t)\right)$ (let $x_i(t):[a,b]\rightarrow \R$ for each $i$)
\begin{itemize}
    \item If $t_0\in[a,b]$, then $\vec{c}^{\prime}\left(t_{0}\right):=\left(x_{1}^{\prime}\left(t_{0}\right), x_{2}^{\prime}\left(t_{0}\right), x_{3}^{\prime}\left(t_{0}\right), \ldots, x_{n}^{\prime}\left(t_{0}\right)\right)$; the velocity vector to $\vec{c}$ at $t_0$
    \item The path $\vec{c}^{\prime}\left(t_{0}\right):=\left(x_{1}^{\prime}\left(t_{0}\right), x_{2}^{\prime}\left(t_{0}\right), x_{3}^{\prime}\left(t_{0}\right), \ldots, x_{n}^{\prime}\left(t_{0}\right)\right)$; the velocity vector to $\vec{c}$ is referred to as velocity of $\vec{c}(t)$
\end{itemize}

Recall chain rule: if $y=f(x)$ where $x$ is a function of $t$, $y'(t)=x'f'(x)$, not to be confused with product rule.
Can write $f'(x)=\frac{y'(t)}{x'(t)}$

\begin{itemize}
    \item If $\vec{p}(t)=\vec{c}(t)+\vec{r}(t),$ then $\vec{p}^{\prime}(t)=\vec{c}^{\prime}(t)+\vec{r}^{\prime}(t)$
    \item If $g(t)=\vec{c}(t) \cdot \vec{r}(t),$ then $g^{\prime}(t)=\vec{c}^{\prime}(t) \cdot \vec{r}(t)+\vec{c}(t) \cdot \vec{r}^{\prime}(t)$
    \item If $\vec{p}(t)=f(t) \vec{c}(t),$ then $\vec{p}^{\prime}(t)=f^{\prime}(t) \vec{c}(t)+f(t) \vec{c}^{\prime}(t)$
    \item If $\vec{p}(t)=\vec{c}(t) \times \vec{r}(t),$ then $\vec{p}^{\prime}(t)=\vec{c}^{\prime}(t) \times \vec{r}(t)+\vec{c}(t) \times \vec{r}^{\prime}(t)$
    \item If $\vec{p}(t)=\vec{c}(f(t)),$ then $\vec{p}^{\prime}(t)=f^{\prime}(t) \vec{c}^{\prime}(f(t))$
    \item If $g(t)=\|\vec{c}(t)\|,$ then $g^{\prime}(t)=\frac{\vec{c}(t) \cdot \vec{c}^{\prime}(t)}{\|\vec{c}(t)\|}$
\end{itemize}

\section{Space Curves}

\begin{itemize}
    \item Projection into the $x y$ plane is the path $(x(t), y(t), 0)$.
    \item Projection into the $x z-$ plane is the path $(x(t), 0, z(t))$.
    \item Projection into the $y z$ plane is the path $(0, y(t), z(t))$.
\end{itemize}

\section{Speed and Arclength}

\subsection{Speed}

Speed of a parametric function in $\R^n$ is given by:

$$||\vec{c}'(t)||=\sqrt{\displaystyle\sum_{i=1}^{n}c_i(t)^2}$$

(being the magnitude of the velocity vector)

\subsection{Arclength}

Arclength of a parametric function is given by:

\[S=\int_a^b ||\vec{c}'(t)||dt=\int_a^b\sqrt{(\frac{dx}{dt})^2+(\frac{dy}{dt})^2+(\frac{dz}{dt})^2+\cdots}\;dt\]

Can approximate arclength as a sum of the lengths of secant vector approximations
$\vec{s}_i=\vec{c}(t_i)-\vec{c}(t_{i-1})$:

\[\mbox{arclength}\approx\sum_{i=1}^n||\vec{s}_i||\]

According to the MVT, there exists a $\hat{t}_i$ in $(t_{i-1},t_i)$ (open interval due to differentiability requirement) such that:

\begin{align*}
    x'(\hat{t}_i)&=\frac{x(t_i)-x(t_{i-1})}{t_i-t_{i-1}}\\
    y'(\hat{t}_i)&=\frac{y(t_i)-y(t_{i-1})}{t_i-t_{i-1}}
\end{align*}

This means that, since $\vec{s}_i$ is given as the difference between 2 points,
being a secant:

\begin{align*}
    \vec{s}_i&=\left((t_i-t_{i-1})x(\hat{t}_i), (t_i-t_{i-1})y(\hat{t}_i)\right )\\
    \vec{s}_i&=(t_i-t_{i-1})\left(x'(\hat{t}_i),y'(\hat{t}_i)\right)\\
    \vec{s}_i&=(t_i-t_{i-1})\vec{c}\,'(\hat{t}_i)
\end{align*}

Thus,

\begin{align*}
    \mbox{arclength}&\approx\sum_{i=1}^n||\vec{s}_i||\\
    \mbox{arclength}&\approx\sum_{i=1}^n||\Delta t \,\vec{c}\,'(\hat{t}_i)||\\
    \mbox{arclength}&\approx\sum_{i=1}^n\Delta t||\vec{c}\,'(\hat{t}_i)||
\end{align*}

Can define the arclength differential as follows:

\[\mathrm{d}s=\sqrt{\mathrm{d}x^2+\mathrm{d}y^2}\]

Can just define arclength as $\text{arclength}=\int \mathrm{ds}$

\subsection{Arclength Parameterization}

Higher the speed of a curve, farther the points are spaced apart.
An arclength parametrization of a curve is a path whose image is the desired curve and whose speed is constantly one.
Or, $\vec{c}:[a,b]\to\mathbb{R}^n$ with $||\vec{c}\,'(t)||=1$ for $t\in[a,b]$.
If a curve is not an arclength parameterization, then can do $\frac{\vec{c}(t)}{||\vec{c}'(t)||}$.\newline

\noindent
When speed is variable, is difficult to define arclength parameterization.
Thus, can define displacement to be $s(t)=\int_a^b (t)dt$. If $v(t)!\neq 0$,
then $s$ is injective because according to FTC, $s'(t)=v(t)$.
By definition, $v'(t)\geq 0$ always since it is composed of a radical, so it must be \textbf{increasing}.
Thus, if $t_1=t_2$, $s(t_1)\neq s(t_2)$.\newline

\noindent
This means that $s$ is invertible, so can solve for $t$ to get $t=\varphi(s)$.
An arclength parameterization can be found by:

\[\boxed{\vec{p}(s)=\vec{c}(\varphi(s))}\]